\section{Módulos}

\begin{frame}[fragile]{Organização do programa em módulos}

    \begin{itemize}
        \item Os \textbf{módulos} são uma importante característica da programação estruturada,
            permitindo o agrupamento lógico de trechos de código semanticamente relacionados

        \item As linguagens de programação que suportam o paradigma estruturado, em geral, permitem
            a separação de módulos em arquivos distintos

        \item Os módulos também podem oferecer controle, completo ou parcial, de acesso às 
            variáveis, funções e subrotinas definidas no módulo

        \item A possibilidade de um módulo importar outros módulos favorece o reuso de código e 
            a construção de bibliotecas de funções e subrotinas

        \item Além disso, no caso de linguagens compiladas, os módulos podem ser pré-compilados,
            acelerando o processo de compilação do programa e facilitando a depuração e manutenção
    \end{itemize}

\end{frame}

\begin{frame}[fragile]{Declaração e implementação de módulos em Fortran}

    \begin{itemize}
        \item A sintaxe para a declaração de um módulo em Fortran é a seguinte:

            \inputsyntax{fortran}{codes/modules.st}
 
        \item O bloco do programa, as funções e as subrotinas podem acessar o módulo por meio
            do comando \code{fortran}{use}:

            \inputsyntax{fortran}{codes/use.st}

        \item A declaração de funções e subrotinas é opcional

        \item Cada módulo deve estar em um arquivo separado 

        \item Os módulos devem ser compilados com a \textit{flag} \texttt{'-c'}

        \item O acesso às variáveis do módulo pode ser controlado por meio dos atributos
            \code{fortran}{private} e \code{fortran}{public}

    \end{itemize}

\end{frame}

\begin{frame}[fragile]{Exemplo de uso de módulos em Fortran}
    \inputsnippet{fortran}{1}{22}{codes/is_prime.f90}
\end{frame}

\begin{frame}[fragile]{Exemplo de uso de módulos em Fortran}
    \inputsnippet{fortran}{1}{22}{codes/primes.f90}
\end{frame}

\begin{frame}[fragile]{Exemplo de uso de módulos em Fortran}
    \inputsnippet{fortran}{23}{44}{codes/primes.f90}
\end{frame}

\begin{frame}[fragile]{Exemplo de uso de módulos em Fortran}
    \inputsnippet{fortran}{45}{66}{codes/primes.f90}
\end{frame}

\begin{frame}[fragile]{Exemplo de uso de módulos em Fortran}
    \inputsnippet{fortran}{67}{89}{codes/primes.f90}
\end{frame}

\section{Fortran}

\begin{frame}[fragile]{Fortran}

    \begin{itemize}
        \item Fortran (\textit{IBM Mathematical FORmula TRANslation System}) é uma linguagem de
            programação desenvolvida na décade de 1950 

        \item Até os dias atuais é uma das principais (ou a principal) linguagem utilizada em
            programação científica

        \item O primeiro compilador foi desenvolvido na IBM, por uma equipe liderada por
            John W. Backus, nos anos de 1954 a 1957

        \item O ISO/IEC 1539-1:1997 contém o padrão Fortran 95, um dos mais populares da linguagem

        \item Fortran apresenta notável performance em computação numérica, o que levou a sua
            adoção em pesquisas científicas e aplicações computacionalmente intensivas, como
            meteorologia, física, engenharia, etc
    \end{itemize}

\end{frame}
\begin{frame}[fragile]{GFortran}

    \begin{itemize}
        \item O projeto GNU Fortran (GFortran) consiste em um \textit{front-end} de compilador e 
            bibliotecas de \textit{run-time} para o GCC que dêem suporte à linguagem Fortran
            
        \item Ele é totalmente compatível com o padrão Fortran 95 e incluí suporte legado ao
            formato Fortran 77

        \item Em distribuições Linux com suporte ao \texttt{apt}, ele pode ser instalado com o
        comando
        \begin{center}
            \verb|$ sudo apt-get install gfortran|
        \end{center}

        \item Para testar a instalação, insira o seguinte comando no terminal:
        \begin{center}
            \verb|$ f95| -v
        \end{center}
    \end{itemize}

\end{frame}

\begin{frame}[fragile]{\it Hello World!}

    \inputcode{fortran}{codes/hello.f90}

\end{frame}

\begin{frame}[fragile]{Compilação, linkedição e execução}

    \begin{itemize}
        \item Para compilar um código Fortran (extensões \texttt{.f90})
            é preciso invocar o GFortran, utilizando a \textit{flag} \texttt{-c}:

        \begin{center}
            \verb|$ f95 -c hello.f90|
        \end{center}
        
        \item No processo de linkedição é preciso indicar, os código-objetos que comporão o 
            executável e, opcionalmente, o nome deste executável (opçaõ \texttt{-o}):

        \begin{center}
            \verb|$ f95 hello.o -o hello|
        \end{center}

        \item É possível executar ambas etapas em um só comando:
        \begin{center}
            \verb|$ f95 hello.f90 -o hello|
        \end{center}

        \item Para rodar o executável criado, basta usar os mesmo mecanismos disponíveis em Linux
            para invocar um programa como, por exemplo, indicar seu caminho:
        
        \begin{center}
            \verb|$ ./hello|
        \end{center}

    \end{itemize}

\end{frame}


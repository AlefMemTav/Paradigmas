\section{Entrada e Saída}

\begin{frame}[fragile]{Saída de dados}

    \begin{itemize}
        \item A função \code{fortran}{write} permite a escrita de uma lista (\texttt{list})
            de dados em um fluxo (\texttt{stream}), de acordo com a formatação dada em 
            \texttt{label}:

            \inputsyntax{fortran}{codes/write.st}

        \item O fluxo pode ser um número associado a um arquivo, uma variável do tipo 
            \code{fortran}{character} ou o símbolo \texttt{*}, que indica o valor padrão (em geral,
            o terminal)

        \item O rótulo (\texttt{label}) é o inteiro identificador do formato, ou \texttt{*} para
            formato livre 

        \item A sintaxe para a declaração de um rótulo é

            \inputsyntax{fortran}{codes/label.st}

        \item Os descritores de formato são uma lista de itens, separados por vírgula, que 
            determinam como a saída deve ser apresentada

    \end{itemize}

\end{frame}

\begin{frame}[fragile]{Exemplos de descritores de formato}

    \begin{table}[ht]
        \centering
        \begin{tabularx}{0.95\textwidth}{lX}
            \toprule
            \textbf{Descritor} & \textbf{Efeito} \\
            \midrule
            $nIw$ & Imprime os próximos $n$ inteiros, com tamanho de $w$ caracteres cada \\
            \rowcolor[gray]{0.9}
            $nFw.d$ & Imprime os próximos $n$ números complexos ou reais, em ponto fixo,
                com $w$ caracteres, e $d$ dígitos na parte decimal \\ 
            $nEw.d$ & Imprime os próximos $n$ números complexos ou reais, em ponto flutuante,
                com $w$ caracteres, e $d$ dígitos na parte decimal \\ 
            \rowcolor[gray]{0.9}
            $Aw$ & Imprime a variável não-numérica $A$, com $w$ caracteres de tamanho \\
            \bottomrule
        \end{tabularx}
    \end{table}

\end{frame}

\section{Introdução}

\begin{frame}[fragile]{Programação Vetorial}

    \begin{itemize}
        \item A \textbf{programação vetorial} (\textit{array programming}) é um paradigma de 
            programação que permite a aplicação de operações em um conjunto de valores de uma só vez
        \pause

        \item É usado predominantemente na programação científica ou em engenharias
        \pause

        \item Linguagens que suportam a programação vetorial são denominadas linguagens 
            \textbf{vetoriais} ou \textbf{multidimensionais}
        \pause

        \item As primitivas destas linguagens expressão concisamente ideias sobre manipulação de 
            dados
        \pause
        \item Não é incomum encontrar códigos de uma só linha em programação vetorial que equivalem
            a códigos de dezenas de linhas em outros paradigmas
        \pause

        \item Exemplos de linguagens que suportam a programação vetorial: APL, J, MATLAB, 
            Mathematica, Octave, R, etc
    \end{itemize}

\end{frame}

\begin{frame}[fragile]{Principais características da programação vetorial}

    \begin{itemize}
        \item Uma vez que as operações agem em coleções de objetos de uma só vez, é possível pensar
            e operar em dados sem uso de laços explícitos ou operações escalares
        \pause

        \item Este paradigma estimula o pensamento dos dados em blocos de elementos correlatos e a
            exploração das propriedades destes elementos
        \pause


        \item As funções são classificadas de acordo com o número de dimensões dos dados sob os 
            quais elas agem
        \pause


        \item Funções escalares (\textit{rank} 0) agem em elementos de dimensão zero. Exemplo: 
            adição nos inteiros
        \pause

        \item Funções vectoriais agem em vetores (dados unidimensionais). Ex.: produto vetorial
        \pause

        \item Funções matriciais, como multiplicação matricial, agem em matrizes (elementos
            bidimensionais, \textit{rank} 2)
    \end{itemize}

\end{frame}

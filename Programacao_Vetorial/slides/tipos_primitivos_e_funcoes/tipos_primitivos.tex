\section{Tipos primitivos de dados}

\begin{frame}[fragile]{Expressões em APL}

    \begin{itemize}
        \item APL pode ser vista como uma notação matemática que também é executável por máquina
        \pause

        \item A linguagem é composta por funções, operadores, \textit{arrays} e atribuições
        \pause

        \item Qualquer código que pode ser aplicado a dados é chamado função 
        \pause

        \item Dois exemplos de funções seriam a adição (\code{apl}{+}) e subtração (\code{apl}{-})
        \pause

        \item As funções de APL podem ser aplicadas monadicamente (prefixada, um operando) ou diadicamente (infixada, dois operando, um à esquerda e outro à direita)
        \pause

        \item O tipo de dados mais elementar é o escalar (\textit{array} de dimensão zero)
    \end{itemize}

\end{frame}

\begin{frame}[fragile]{Inteiros}

    \begin{itemize}
        \item Números são tratados internamente pela APL quanto ao tamanho e tipo e podem ser misturados sem problemas
        \pause

        \item Em APL os números podem ser inteiros, reais (em ponto flutuante) e números complexos
        \pause

        \item Um escalar inteiro pode ser grafado usando a notação decimal padrão:
            \inputsyntax{apl}{codes/int.apl}
        \pause

        \item Comentários são precedidos pelo símbolo \code{apl}{⍝}
        \pause

        \item Números negativos são precedidos pelo símbolo \code{apl}{¯} (\textit{macron})

            \inputsyntax{apl}{codes/sub.apl}
    \end{itemize}

\end{frame}


\begin{frame}[fragile]{Novo símbolo}

   \newAPLsymbol{+}{plus}{diádico}{Adição escalar}{U+002B}{-}{-} 

\end{frame}

\begin{frame}[fragile]{Novo símbolo}

   \newAPLsymbol{-}{minus}{diádico}{Subtração escalar}{U+002D}{-}{-} 

\end{frame}

\begin{frame}[fragile]{Novo símbolo}

   \newAPLsymbol{×}{times}{diádico}{Multiplicação escalar}{U+00D7}{x x <tab>}{APL + -} 

\end{frame}

\begin{frame}[fragile]{Novo símbolo}

   \newAPLsymbol{¯}{macron}{monádico}{Antecede um número negativo}{U+00AF}{- - <tab>}{APL + 2} 

\end{frame}

\begin{frame}[fragile]{Novo símbolo}

   \newAPLsymbol{⍝}{comment}{monádico}{Inicia um comentário. Tudo que o sucede até o fim da linha será considerado comentário}{U+235D}{o n <tab>}{APL + ,} 

\end{frame}

\begin{frame}[fragile]{Números reais}

    \begin{itemize}
        \item Em escalares reais, a parte inteira é separada das casas decimais por meio do ponto final
            \inputsyntax{apl}{codes/float.apl}
        \pause

        \item APL também trata problemas de precisão de forma transparente ao usuário
            \inputsyntax{apl}{codes/precision.apl}
        \pause

        \item O símbolo \code{apl}{⋄} (\textit{diamond}) separa duas expressões em uma mesma linha
        \pause

        \item Notação científica pode representar números muito pequenos ou grandes
            \inputsyntax{apl}{codes/scientific.apl}

   \end{itemize}

\end{frame}

\begin{frame}[fragile]{Novo símbolo}

   \newAPLsymbol{÷}{divide}{diádico}{Divisão escalar. Divisão por zero resulta em um erro}{U+00F7}{: - <tab>}{APL + =} 

\end{frame}

\begin{frame}[fragile]{Novo símbolo}

   \newAPLsymbol{⋄}{diamond}{diádico}{Separador de expressões}{U+22C4}{< > <tab>}{APL + '} 

\end{frame}

\begin{frame}[fragile]{Constantes booleanas}

    \begin{itemize}
        \item Em APL: falso é igual a \code{apl}{0} (zero) e verdadeiro é igual a \code{apl}{1} (um)
            \inputsyntax{apl}{codes/bool.apl}
        \pause

        \item Os operadores relacionais retornam valores booleanos
            \inputsyntax{apl}{codes/relational.apl}
    \end{itemize}

\end{frame}

\begin{frame}[fragile]{Novo símbolo}

   \newAPLsymbol{=}{equal}{diádico}{Igual a}{U+003D}{-}{APL + 5} 

\end{frame}

\begin{frame}[fragile]{Novo símbolo}

   \newAPLsymbol{≠}{not equal}{diádico}{Diferente de}{U+2260}{= / <tab>}{APL + 8} 

\end{frame}

\begin{frame}[fragile]{Novo símbolo}

   \newAPLsymbol{<}{less than}{diádico}{Menor que}{U+003C}{-}{APL + 3} 

\end{frame}

\begin{frame}[fragile]{Novo símbolo}

   \newAPLsymbol{>}{greater than}{diádico}{Maior que}{U+003E}{-}{APL + 7} 

\end{frame}

\begin{frame}[fragile]{Novo símbolo}

   \newAPLsymbol{≤}{less than or equal to}{diádico}{Menor ou igual a}{U+2264}{< = <tab>}{APL + 4} 

\end{frame}

\begin{frame}[fragile]{Novo símbolo}

   \newAPLsymbol{≥}{greater than or equal to}{diádico}{Maior ou igual a}{U+2265}{> = <tab>}{APL + 6} 

\end{frame}

\begin{frame}[fragile]{Números complexos}

    \begin{itemize}
        \item O caractere \code{apl}{'J'} separa a parte real da parte imaginária em números complexos
            \inputsyntax{apl}{codes/complex.apl}
        \pause
        \item Lembre-se de que o argumento à direita de uma função diádica é o resultado de toda a expressão à direita do símbolo
            \inputsyntax{apl}{codes/order.apl}
    \end{itemize}

\end{frame}

\begin{frame}[fragile]{Novo símbolo}

   \newAPLsymbol{*}{power}{diádico}{Eleva o argumento à esquerda a potência indicada no argumento à direita}{U+002A}{-}{APL + p} 

\end{frame}

\begin{frame}[fragile]{Caracteres e strings}

    \begin{itemize}
        \item Em APL, strings são vetores de caracteres
        \pause

        \item Tanto caracteres quanto strings são delimitadas por aspas simples
            \inputsyntax{apl}{codes/strings.apl}
        \pause
 
        \item Atribuições podem ser feitas por meio do símbolo \code{apl}{←}
            \inputsyntax{apl}{codes/assignment.apl}
    \end{itemize}

\end{frame}

\begin{frame}[fragile]{Novo símbolo}

   \newAPLsymbol{←}{assign}{diádico}{Atribui o argumento à direta ao argumento à esquerda}{U+2190}{< - <tab>}{APL + '} 

\end{frame}

\begin{frame}[fragile]{Funções aritméticas monádicas}

    \begin{itemize}
        \item As funções aritméticas apresentadas até o momento tem versões monádicas
            \inputsyntax{apl}{codes/arithmetic.apl}
        \pause

        \item Quando aplicada a números reais, a função monádica \code{apl}{×} corresponde à função
            \texttt{signum()} de muitas linguagens
    \end{itemize}

\end{frame}

\begin{frame}[fragile]{Novo símbolo}

   \newAPLsymbol{+}{conjugate}{monádico}{Conjugado complexo}{U+002B}{-}{-} 

\end{frame}

\begin{frame}[fragile]{Novo símbolo}

   \newAPLsymbol{-}{negate}{monádico}{Simétrico aditivo}{U+002D}{-}{-} 

\end{frame}

\begin{frame}[fragile]{Novo símbolo}

   \newAPLsymbol{×}{direction}{monádico}{Vetor unitário na direção do número}{U+00D7}{x x <tab>}{APL + -} 

\end{frame}

\begin{frame}[fragile]{Novo símbolo}

   \newAPLsymbol{÷}{reciprocal}{monádico}{Inverso multiplicativo}{U+00F7}{: - <tab>}{APL + =} 

\end{frame}

\begin{frame}[fragile]{Novo símbolo}

   \newAPLsymbol{*}{exponential}{monádico}{$e$ elevado ao argumento à direita}{U+002A}{-}{APL + p} 

\end{frame}


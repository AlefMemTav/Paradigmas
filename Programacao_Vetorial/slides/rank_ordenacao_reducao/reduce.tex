\section{Reduções e varreduras}


\begin{frame}[fragile]{Reduções}

    \begin{itemize}
        \item Em APL, um operador recebe um ou dois operandos (geralmente funções) como argumentos 
            e retorna uma função (monádica ou diádica)
        \pause

        \item O operador \code{apl}{/} (\textit{reduce}) é monádico e retorna uma função 
            ambivalente (que pode ser usada de forma monádica ou diádica)
        \pause

        \item O nome diz respeito ao fato de que a função resultante reduz o \textit{rank} do 
            argumento em 1 unidade
        \pause

        \item A redução de um vetor é direta: \code{apl}{F/a b c d e ...} equivale a 
            \code{apl}{⊂ a F b F c F d F e F ...}, onde  o resultado tem a mesma forma do argumento, 
            exceto o último eixo
                \inputsyntax{apl}{codes/reduce.apl}

    \end{itemize}

\end{frame}

\begin{frame}[fragile]{Novo símbolo}

   \newAPLsymbol{/}{reduce}{operador}{\code{apl}{F/v1 v2 v3 ... ≡ ⊂ v1 F v2 F v3 F ...}}{U+002F}{-}{-} 

\end{frame}

\begin{frame}[fragile]{Características da redução}

    \begin{itemize}
        \item A ordem de precedência das funções em APL (associativa à direita) afeta o comportamento da redução:
                \inputsyntax{apl}{codes/alternatesum.apl}
        \pause

        \item Em \textit{arrays} com \textit{rank} maior do que um, a redução se aplica sempre na última dimensão
                \inputsyntax{apl}{codes/reduce2.apl}
    \end{itemize}

\end{frame}

\begin{frame}[fragile]{Novo símbolo}

   \newAPLsymbol{⍟}{logarithm}{diádico}{Computa $\log_\alpha \omega$}{U+235F}{* O <tab>}{APL + Shift + *} 

\end{frame}

\begin{frame}[fragile]{Novo símbolo}

   \newAPLsymbol{⍟}{logarithm}{monádico}{Computa $\ln \omega$}{U+235F}{* O <tab>}{APL + Shift + *} 

\end{frame}


\begin{frame}[fragile]{Redução na demais dimensões}

    \begin{itemize}
        \item  O operador \code{apl}{⌿} (\textit{reduce first}) produz uma redução em relação à primeira dimensão
                \inputsyntax{apl}{codes/reducefirst.apl}
        \pause
        
        \item Para produzir uma redução em relação a $k$-ésima dimensão, use a notação \code{apl}{f/[k]}
                \inputsyntax{apl}{codes/reducek.apl}
        \pause

        \item A notação \code{apl}{f[1]} equivale a \code{apl}{f⌿}
    \end{itemize}

\end{frame}

\begin{frame}[fragile]{Novo símbolo}

   \newAPLsymbol{⌿}{reduce first}{operador}{Produz um redução em relação a primeira dimensão}{U+233F}{/ - <tab>}{APL + ;} 

\end{frame}

\begin{frame}[fragile]{Reduções diádicas}

    \begin{itemize}
        \item Na forma diádica, \code{apl}{L f/ R} é uma redução que usa uma janela de tamanho
            \code{apl}{L} em \code{apl}{R}
        \pause

        \item Esta redução por janela não altera o \textit{rank} do argumento
                \inputsyntax{apl}{codes/dyadicreduce.apl}
        \pause

        \item Se \code{apl}{L} é negativo, a janela é invertida    
                \inputsyntax{apl}{codes/inverted.apl}
    \end{itemize}

\end{frame}

\begin{frame}[fragile]{Funções lógicas e reduções}

    \begin{itemize}
        \item A função \code{apl}{all} retorna \code{apl}{1} se todos os booleanos do argumento são iguais a 1
                \inputsyntax{apl}{codes/all.apl}
        \pause

        \item A função \code{apl}{any} retorna \code{apl}{1} se ao menos um booleano do argumento é igual a 1
                \inputsyntax{apl}{codes/any.apl}
        \pause

        \item De fato, a função diádica \code{apl}{∧} computa o mínimo múltiplo comum de seus argumentos e equivale a conjunção lógica para valores booleanos
        \pause

        \item A função diádica \code{apl}{∨} computa o maior divisor comum de seus argumentos, assumindo que $\gcd(0, 0) = 0$, de forma que equivale a disjunção lógica para valores booleanos
    \end{itemize}
\end{frame}

\begin{frame}[fragile]{Novo símbolo}

   \newAPLsymbol{∧}{lcm}{diádico}{Computa o mínimo múltiplo comum entre \code{apl}{⍺} e 
        \code{apl}{⍵}}{U+2227}{\^{} \^{} <tab>}{APL + 0} 

\end{frame}

\begin{frame}[fragile]{Novo símbolo}

   \newAPLsymbol{∨}{gcd}{diádico}{Computa o maior divisor comum entre \code{apl}{⍺} e 
        \code{apl}{⍵}}{U+2228}{v v <tab>}{APL + 9} 
\end{frame}

\begin{frame}[fragile]{Fatorial}

    \begin{itemize}
        \item É possível computar o fatorial de um inteiro positivo $n$  de, no mínimo, três formas
            distintas em APL
        \pause

        \item A primeira delas é recorrer a função monádica \code{apl}{!}, a segunda é 
            implementar uma \textit{dfn} recursiva e a terceira é usar uma redução
        \pause

        \item A função \code{apl}{cmpx} do \textit{workspace} \textit{dfns} pode ser usada para
            comparar a performance destas três implementações
                \inputsyntax{apl}{codes/factorial.apl}
    \end{itemize}

\end{frame}

\begin{frame}[fragile]{Varredura}

    \begin{itemize}
        \item O operador \inputmintinline{apl}{codes/op.apl} (\textit{scan}) gera uma função monádica que age nos
            prefixos das últimas dimensões de seu parâmetro
        \pause

        \item A expressão \inputmintinline{apl}{codes/expr.apl} equivale a \code{apl}{(f/a) (f/a b) (f/a b c) ...}
                \inputsyntax{apl}{codes/scan.apl}
        \pause

        \item O operador \code{apl}{⍀} gera uma varredura na primeira dimensão
    \end{itemize}
\end{frame}

\begin{frame}[fragile]{Novo símbolo}
   \newAPLsymbol{\textbackslash}{scan}{operador}{Gera uma varredura na última dimensão}{U+005C}{-}{-} 
\end{frame}

\begin{frame}[fragile]{Novo símbolo}

   \newAPLsymbol{⍀}{scan first}{operador}{Gera uma varredura na primeira dimensão}{U+2340}{\textbackslash\ - <tab>}{APL + .} 
\end{frame}

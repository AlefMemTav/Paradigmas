\section{Manipulação de arquivos}

\begin{frame}[fragile]{I/O em arquivos}

    \begin{itemize}
        \item A biblioteca \code{haskell}{System.IO} fornece uma série de funções para I/O, 
            inclusive em arquivos

        \item A função \code{haskell}{openFile()} retorna um \code{haskell}{Handle} para o arquivo
            aberto, se bem sucedida

            \inputsyntax{haskell}{codes/openfile.hs}

        \item Os modos de abertura são \code{haskell}{ReadMode, WriteMode, AppendMode} e
            \code{haskell}{ReadWriteMode}

        \item Uma vez finalizada a manipulação do arquivo, ele deve ser fechado por meio da
            função \code{haskell}{hClose()}

        \item As funções \code{haskell}{hPutStrLn()} e \code{haskell}{hGetLine()} funcionam 
            como as contrapartes sem o \code{haskell}{h}, com a diferença que agem sobre o
            arquivo indicado
    \end{itemize}

\end{frame}

\begin{frame}[fragile]{Exemplo de manipulação de arquivos}
    \inputsnippet{haskell}{1}{21}{codes/stripFile.hs}
\end{frame}

\begin{frame}[fragile]{\code{haskell}{return}}

    \begin{itemize}
        \item Em Haskell, a palavra reservada \code{haskell}{return} tem significado distinto do
            utilizado em outras linguagens, como C/C++

        \item \code{haskell}{return} insere um valor em uma ação de I/O

        \item Ele corresponde ao inverso do operador \code{haskell}{<-}

        \item Deste modo, ele não encerra prematuramente um bloco, como nas linguagens imperativas,
            e sim produz uma ação de I/O
    \end{itemize}

\end{frame}

\begin{frame}[fragile]{Controlando o cursor de leitura}

    \begin{itemize}
        \item Quando uma rotina de leitura é chamada em um \textit{handle} de um arquivo, o cursor
            de leitura é atualizado

        \item Assim, a próxima leitura tem início no ponto no qual a leitura anterior terminou

        \item A função \code{haskell}{hTell()} retorna a posição atual do cursor no arquivo,
            contada em número de \textit{bytes}:

            \inputsyntax{haskell}{codes/htell.hs}

        \item Ao abrir o arquivo, o cursor inicia na posição 0, exceto no
            \code{haskell}{AppendMode}, onde o cursor inicia no fim do arquivo

        \item A função \code{haskell}{hSeek()} permite atualizar a posição do cursor:

            \inputsyntax{haskell}{codes/hseek.hs}
            
        \item Os modos disponíveis são: \code{haskell}{AbsoluteSeek, RelativeSeek} e
            \code{haskell}{SeekFromEnd}
    \end{itemize}

\end{frame}

\begin{frame}[fragile]{Exemplo de controle do cursor de leitura}
    \inputsnippet{haskell}{1}{21}{codes/filesize.hs}
\end{frame}


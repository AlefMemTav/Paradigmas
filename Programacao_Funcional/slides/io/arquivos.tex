\section{Manipulação de arquivos}

\begin{frame}[fragile]{I/O em arquivos}

    \begin{itemize}
        \item A biblioteca \code{haskell}{System.IO} fornece uma série de funções para I/O, 
            inclusive em arquivos

        \item A função \code{haskell}{openFile()} retorna um \code{haskell}{Handle} para o arquivo
            aberto, se bem sucedida

            \inputsyntax{haskell}{codes/openfile.hs}

        \item Os modos de abertura são \code{haskell}{ReadMode, WriteMode, AppendMode} e
            \code{haskell}{ReadWriteMode}

        \item Uma vez finalizada a manipulação do arquivo, ele deve ser fechado por meio da
            função \code{haskell}{hClose()}

        \item As funções \code{haskell}{hPutStrLn()} e \code{haskell}{hGetLine()} funcionam 
            como as contrapartes sem o \code{haskell}{h}, com a diferença que agem sobre o
            arquivo indicado
    \end{itemize}

\end{frame}

\begin{frame}[fragile]{Exemplo de manipulação de arquivos}
    \inputsnippet{haskell}{1}{18}{codes/stripFile.hs}
\end{frame}

\begin{frame}[fragile]{\code{haskell}{return}}

    \begin{itemize}
        \item Em Haskell, a palavra reservada \code{haskell}{return} tem significado distinto do
            utilizado em outras linguagens, como C/C++

        \item \code{haskell}{return} insere um valor em uma ação de I/O

        \item Ele corresponde ao inverso do operador \code{haskell}{<-}

        \item Deste modo, ele não encerra prematuramente um bloco, como nas linguagens imperativas,
            e sim produz uma ação de I/O
    \end{itemize}

\end{frame}

\begin{frame}[fragile]{Controlando o cursor de leitura}

    \begin{itemize}
        \item Quando uma rotina de leitura é chamada em um \textit{handle} de um arquivo, o cursor
            de leitura é atualizado

        \item Assim, a próxima leitura tem início no ponto no qual a leitura anterior terminou

        \item A função \code{haskell}{hTell()} retorna a posição atual do cursor no arquivo,
            contada em número de \textit{bytes}:

            \inputsyntax{haskell}{codes/htell.hs}

        \item Ao abrir o arquivo, o cursor inicia na posição 0, exceto no
            \code{haskell}{AppendMode}, onde o cursor inicia no fim do arquivo

        \item A função \code{haskell}{hSeek()} permite atualizar a posição do cursor:

            \inputsyntax{haskell}{codes/hseek.hs}
            
        \item Os modos disponíveis são: \code{haskell}{AbsoluteSeek, RelativeSeek} e
            \code{haskell}{SeekFromEnd}
    \end{itemize}

\end{frame}

\begin{frame}[fragile]{Exemplo de controle do cursor de leitura}
    \inputsnippet{haskell}{1}{21}{codes/filesize.hs}
\end{frame}

\begin{frame}[fragile]{Entrada e saída padrão do sistema}

    \begin{itemize}
        \item Em Haskell há três \textit{handles} pré-definidos, associados a entrada e a
            saída-padrão do sistema

        \item O \textit{handle} \code{haskell}{stdin} corresponde à entrada padrão, que usualmente
            está associada ao teclado

        \item O \textit{handle} \code{haskell}{stdout} corresponde à saída padrão, em geral 
            associada ao terminal

        \item Por fim o \textit{handle} \code{haskell}{stderr} corresponde à saída de erro padrão,
            que pode ser o próprio terminal ou ser direcionada ao um arquivo específico

        \item As funções de I/O sem o prefixo \code{haskell}{h} são definidas em termos destes
            \code{haskell}{handles}:

            \inputsyntax{haskell}{codes/io.hs}

    \end{itemize}

\end{frame}

\begin{frame}[fragile]{Removendo e renomeando arquivos}

    \begin{itemize}
        \item A biblioteca \code{haskell}{System.Directory} oferece funções que permitem
            manipular os arquivos diretamente

        \item A função \code{haskell}{removeFile} recebe como argumento o caminho para um arquivo
            e o remove

            \inputsyntax{haskell}{codes/removefile.hs}

        \item Já a função \code{haskell}{renameFile} recebe dois parâmetros: o caminho do arquivo
            a ser renomeado e o caminho do seu novo nome/destino

            \inputsyntax{haskell}{codes/renamefile.hs}

        \item Se o arquivo de destino já existir, ele será removido para que então o arquivo
            original assuma seu caminho/nome
    \end{itemize}

\end{frame}

\begin{frame}[fragile]{Exemplo de manipulação de nome de arquivo}
    \inputsnippet{haskell}{1}{21}{codes/normalize.hs}
\end{frame}

\begin{frame}[fragile]{\textit{Lazy} I/O}

    \begin{itemize}
        \item Haskell também oferece funções para utilizar o \textit{lazy evaluation} no contexto
            de I/O

        \item A função \code{haskell}{hGetContents} recebe um \textit{handle} e retorna uma
            string com o todo conteúdo a partir do ponto onde está localizado o cursor de
            leitura:

            \inputsyntax{haskell}{codes/hgetcontents.hs}

        \item Contudo, o acesso a este conteúdo é feito por demanda: a string não contém, 
            necessariamente, todo o conteúdo de uma só vez em memória

        \item Este acesso é feito de forma transparente ao usuário que, para fins de código,
            terá uma string que pode ser utilizada em código puro

        \item  Assim é possível escrever código com I/O mais próximos da abordagem funcional,
            sem a necessidade de laços para processamento de blocos de informações, como é feito
            nas linguagens imperativas
    \end{itemize}

\end{frame}

\begin{frame}[fragile]{Exemplo de I/O no estilo funcional}
    \inputsnippet{haskell}{1}{21}{codes/stripFileLazy.hs}
\end{frame}

\begin{frame}[fragile]{\textit{Lazy} I/O}

    \begin{itemize}
        \item O idioma de abrir um arquivo, ler seu conteúdo, processar este conteúdo, escrever
            o resultado em um arquivo de saída e fechar os \textit{handles} é tão comum que
            Haskell oferece duas funções que abstraem este processo

        \item A primeira delas é a função \code{haskell}{readFile}, que recebe o caminho do
            arquivo a ser aberto e retorna uma string acessa o conteúdo por meio de
            \textit{lazy evaluation}

            \inputsyntax{haskell}{codes/readfile.hs}

        \item A segunda é a função \code{haskell}{writeFile}, que recebe o caminho do arquivo de
            saída e o conteúdo a ser escrito:

            \inputsyntax{haskell}{codes/writefile.hs}

        \item Ambas funções fazem parte da biblioteca \code{haskell}{System.IO}

        \item Estas funções mantém o \textit{handle} dos arquivos internamente, dispensando a
            chamada da função \code{haskell}{hClose}
    \end{itemize}

\end{frame}

\begin{frame}[fragile]{Exemplo de I/O no estilo funcional sem \textit{handles}}
    \inputsnippet{haskell}{1}{21}{codes/stripfile.hs}
\end{frame}

\begin{frame}[fragile]{Interações}

    \begin{itemize}
        \item Quando os arquivos a serem processados são a entrada e a saída padrão, o idioma
            abstraído pelas funções \code{haskell}{readFile} e \code{haskell}{writeFile}
            pode ser condensado em uma única função

        \item A função \code{haskell}{interact} tem como parâmetro uma função que recebe e retorna
            strings:

            \inputsyntax{haskell}{codes/interact.hs}

        \item Todo o conteúdo da entrada padrão é passado como parâmetro desta função, e o retorno
            dela é escrito na saída padrão

        \item Esta função também faz parte da biblioteca \code{haskell}{System.IO}

        \item Quando utilizada em conjunto com filtros, esta função permite escrever programas
            concisos, com poucas linhas, que processam arquivos de diferentes maneiras
    \end{itemize}

\end{frame}

\begin{frame}[fragile]{Exemplo de interação com filtro}
    \inputsnippet{haskell}{1}{21}{codes/stripcomments.hs}
\end{frame}


\section{Entrada e saída em console}

\begin{frame}[fragile]{Código puro e código impuro}

    \begin{itemize}
        \item Haskell determina uma clara separação entre código puro e código impuro

        \item Esta estratégia permite que código puro fique isento de efeitos colaterais

        \item Além de facilitar a divisão semântica do código, ela permite aos compiladores
            otimizar e paralelizar trechos de código automaticamente

        \item Como as rotinas de entrada e saída interagem com o mundo externo, todas elas
            produzem ou estão suscetíveis a efeitos colaterais, sendo assim, códigos impuros
    \end{itemize}

\end{frame}

\begin{frame}[fragile]{Leitura e escrita de strings em console}

    \begin{itemize}
        \item Haskell provê um conjunto de funções para escrita e leitura de dados a partir 
            do console

        \item No que diz respeito à strings, duas funções básicas são \code{haskell}{putStrLn}
            e \code{smalltalk}{getLine}

        \item A função \code{haskell}{putStrLn} escreve uma string no console, seguida de uma
            quebra de linha:

            \inputsyntax{haskell}{codes/putstrln.hs}

        \item Já a função \code{haskell}{getLine} lê uma string do console até encontrar uma 
            quebra de linha e a retorna, sem a quebra

        \item O tipo da função \code{haskell}{getLine} é

            \inputsyntax{haskell}{codes/getline.hs}
    \end{itemize}

\end{frame}

\begin{frame}[fragile]{Exemplo de leitura e escrita de strings em console}
    \inputcode{haskell}{codes/echo.hs}
\end{frame}

\begin{frame}[fragile]{Ações de entrada e saída}

    \begin{itemize}
        \item O identificador \code{haskell}{IO} nas assinaturas das funções
            \code{haskell}{getLine()} e \code{haskell}{putStrLn()} indicam que estas funções
            ou tem efeitos colaterais, ou podem retornar valores distintos mesmo que invocadas
            com os mesmos parâmetros

        \item Portanto, a presença de \code{haskell}{IO} na assinatura das funções indicam 
            código impuro

        \item Uma ação de entrada e saída (I/O) tem tipo \code{haskell}{IO} \textit{tipo}

        \item Uma ação pode ser declarada e armazenada, mas só pode ser executada dentro de outra
            ação de I/O

        \item \code{haskell}{()} é a tupla vazia e indica a ausência de retorno (similar ao tipo
            \code{cpp}{void} de C/C++)

    \end{itemize}

\end{frame}

\begin{frame}[fragile]{Ações de entrada e saída}

    \begin{itemize}
        \item Assim, a função \code{haskell}{putStrLn()} não tem retorno: escrever a string no
            terminal é seu efeito colateral

        \item Executar uma ação do tipo \code{haskell}{IO} \code{haskell}{t} pode gerar algum tipo
            de I/O e, eventualmente, retornar um valor do tipo \code{haskell}{t}

        \item Já a função \code{haskell}{getLine()} retorna uma ação do tipo \code{haskell}{IO
            String}

        \item O operator \code{haskell}{<-} executa a ação e extrai o retorno do ação, unindo-o
            a variável indicada

        \item A função \code{haskell}{main()}, ponto de início da execução dos programas em
            Haskell, tem tipo \code{haskell}{IO ()}

        \item A palavra reservada \code{haskell}{do} define uma sequência de ações de I/O

        \item O valor do bloco \code{haskell}{do} é dado pelo valor da última ação executada

        \item A palavra reservada \code{haskell}{let} permite armazenar o resultado de código
            puro em uma acão (ou bloco de ações) de I/O
    \end{itemize}

\end{frame}

\begin{frame}[fragile]{Exemplo de código puro em um bloco de I/O}
    \inputsnippet{haskell}{1}{20}{codes/greetings.hs}
\end{frame}

\begin{frame}[fragile]{Código Puro vs. Código Impuro}

    \begin{itemize}
        \item Haskell distingue e separa os trechos de código puro e impuro

        \item Código puro não tem efeitos colaterais, não altera estados e tem mesmo retorno
            para o mesmo conjunto de parâmetros

        \item Código impuro não tem garantias sobre o retorno, pode interagir com o sistema,
            alterando seu estado, e ter efeitos colaterais

        \item Código impuro pode invocar código puro

        \item Já o contrário não é possível

        \item As garantias dadas por um código puro tem vantanges, como a possibilidade de
            paralelismo automático, por exemplo
    \end{itemize}

\end{frame}

\section{Funções anônimas}

\begin{frame}[fragile]{Funções lambda}

    \begin{itemize}
        \item Haskell permite a definição de funções anônimas, denominadas \textbf{funções lambda}

        \item A sintaxe para a definição de uma função lambda é

            \inputsyntax{haskell}{codes/lambda.hs}

        \item A lista de parâmetros \texttt{var1, var2, ..., varN} pode conter casamentos de
            padrões

        \item A expressão, contudo, não pode conter guardas

        \item A depender do contexto, pode ser necessário usar parêntesis para delimitar o
            corpo da função lambda

        \item Por exemplo, a função abaixo imprime os inteiros de 1 a $n$, substituindo os 
            múltiplos de $m$ pela palavra \code{haskell}{"Pim"}

            \inputsyntax{haskell}{codes/pim.hs}
    \end{itemize}

\end{frame}

\begin{frame}[fragile]{\it Currying}

    \begin{itemize}
        \item A aplicação parcial de uma função (\textit{currying}, termo derivado do nome
            do lógico Haskell Curry) é uma técnica de programação
            funcional que permite obter uma nova função através da aplicação incompleta de
            seus parâmetros

        \item Na sintaxe de tipo de uma função, as setas (\texttt{->}) indicam a sequência de
            aplicações parciais

        \item Por exemplo, o tipo da função \code{haskell}{take} é

            \inputsyntax{haskell}{codes/take.hs}

        \item Uma leitura possível deste tipo seria: ``a função \code{haskell}{take} recebe dois
            parâmetros -- um inteiro e uma lista de elementos do tipo \code{haskell}{a}''

        \item Porém, efetivamente este tipo significa que a função \code{haskell}{take}, ao receber
            como parâmetro um número inteiro, retorna uma função \code{haskell}{f} cujo tipo
            é \code{haskell}{[a] -> [a]}

    \end{itemize}

\end{frame}

\begin{frame}[fragile]{\it Currying}

    \begin{itemize}
        \item Por exemplo, 

            \inputsyntax{haskell}{codes/take3.hs}

        \item De fato, em Haskell, todas as funções recebem um único parâmetro

        \item A cada aplicação de um parâmetro, o retorno é uma nova função, que recebe 
            ``um parâmetro a menos'' que a anterior

        \item Outra forma de interpretar este comportamento é pensar que, se $f$ é uma 
            função com múltiplos parâmetros, a cada aplicação um dos parâmetros tem seu
            valor fixado, e os parâmetros ainda não definidos são os parâmetros da função
            resultante
    \end{itemize}

\end{frame}

\begin{frame}[fragile]{\it Currying}

    \begin{itemize}
        \item A aplicação parcial permite a definição de novas funções baseadas em funções
            preexistentes, simplificando o código e facilitando a leitura do mesmo

        \item Por exemplo, a função \code{haskell}{head} pode ser definida em termos de uma
            aplicação parcial de \code{haskell}{take}

            \inputsyntax{haskell}{codes/head.hs}

        \item De fato, o parâmetro \code{haskell}{xs} pode ser omitido desta definição,
            resultando em

            \inputsyntax{haskell}{codes/head_xs.hs}

        \item Outro exemplo de função definida por meio de aplicação parcial:

            \inputsyntax{haskell}{codes/odd.hs}
    \end{itemize}

\end{frame}

\begin{frame}[fragile]{Seções}

    \begin{itemize}
        \item Haskell tem uma sintaxe especial para aplicação parcial de funções em notação
            infixada, denominada \code{haskell}{seção}

        \item Para tal, basta envolver o operador entre parêntesis, e fornecer o operador da
            esquerda ou da direita 

            \inputsyntax{haskell}{codes/double.hs}
    \end{itemize}

\end{frame}

\begin{frame}[fragile]{Padrão `como'}

    \begin{itemize}
        \item Ao escrever funções que recebem listas como parâmetros, é comum checar dois
            padrões: a lista vazia (\code{haskell}{[]}) e a lista com ao menos um elemento
            (padrão \code{haskell}{(x:xs)})

        \item Este segundo padrão desconstrói a lista, de modo que se a lista toda for necessária
            na expressão que se segue, ela precisa se reconstruída

        \item O padrão `\textbf{como}' (\textit{as-pattern}) permite checar este segundo caso, 
            preservando
            a lista original, de modo que ela pode ser utilizada na expressão diretamente, evitando
            novas reconstruções

        \item Por exemplo, a função abaixo lista todos os sufixos não nulos de uma string dada:

            \inputsyntax{haskell}{codes/suffixes.hs}

    \end{itemize}

\end{frame}

\begin{frame}[fragile]{Composição de funções}

    \begin{itemize}
        \item Algumas funções podem ser implementadas através de uma cadeia de chamadas de
            funções, onde o resultado da aplicação da função anterior ao parâmetro da chamada
            é o parâmetro de entrada da próxima função da cadeia

        \item Por exemplo, a função \code{haskell}{tails} do módulo \code{haskell}{Data.List}
            tem comportamento quase idêntico ao da função \code{haskell}{suffixes} implementada
            anteriormente:

            \inputsyntax{haskell}{codes/tails.hs}

        \item Assim, a função \code{haskell}{suffixes} pode ser implementada utilizando-se as
            funções \code{haskell}{tails} e \code{haskell}{init}:

            \inputsyntax{haskell}{codes/suffixes2.hs}

        \item Este padrão corresponde à composição de funções em matemática

    \end{itemize}

\end{frame}

\begin{frame}[fragile]{Composição de funções}

    \begin{itemize}
        \item Em Haskell, funções podem ser compostas por meio do operador ponto final
            (`\code{haskell}{.}')

        \item De fato, ele é um operador como os demais operadores da linguagens, associativo
            à esquerda, como nível 9 de precedência:

            \inputsyntax{haskell}{codes/dot.hs}
        \item Assim, a função \code{haskell}{suffixes} pode ser implementada como

            \inputsyntax{haskell}{codes/suffixes3.hs}

        \item A função abaixo contabiliza o número de palavras de uma string que começam em
            maiúscula:

            \inputsyntax{haskell}{codes/capCount.hs}

    \end{itemize}

\end{frame}

\section{Mapas, filtros e reduções}

\begin{frame}[fragile]{Laços em Haskell}

    \begin{itemize}
        \item Diferentemente das linguagens imperativas, Haskell não oferece construtos equivalentes
            aos laços \code{c}{for} e \code{c}{while} das linguagens imperativas

        \item Para contornar este fato pode-se valer de algumas técnicas distintas

        \item Uma maneira é utilizar recursão

        \item Outra forma é utilizar funções de alta ordem e abstrações

        \item Esta diferença de abordagem tende a ser um fator que dificulta a 
            aprendizagem de Haskell, e linguagens funcionais em geral, para programadores
            acostumados com linguagens imperativas

    \end{itemize}

\end{frame}

\begin{frame}[fragile]{Exemplo de uso de recursão: Capitalização}

    \begin{itemize}
        \item O primeiro exemplo de uso de recursão para substituir laços é o problema de
            capitalizar as palavras de uma string dada

        \item Por capitalizar uma palavra entende-se:
        \begin{enumerate}
            \item tornar a primeira letra maiúscula; e
            \item transformar todas as demais em minúsculas
        \end{enumerate}

        \item Para ilustrar as diferenças entre as abordagens imperativa e funcional, será
            apresentado um código C++ que capitaliza strings

        \item Em seguida, será apresentado um código equivalente em Haskell, utilizando
            recursão em substituição ao laço

        \item Para focar apenas no processo de capitalização, o parâmetro das funções será uma
            lista de palavras, de modo a abstração o processo de tokenização
    \end{itemize}

\end{frame}

\begin{frame}[fragile]{Implementação da capitalização em C++}
    \inputsnippet{c++}{5}{25}{codes/capitalize.cpp}
\end{frame}

\begin{frame}[fragile]{Implementação da capitalização em Haskell}
    \inputcode{haskell}{codes/capitalize.hs}
\end{frame}

\begin{frame}[fragile]{Exemplo de uso de recursão: Lista de Aprovados}

    \begin{itemize}
        \item Um segundo exemplo de uso de recursão em substituição a laços é o de gerar
            uma lista de aprovados, a partir de uma lista de alunos e suas menções

        \item O critério de aprovação é ter nota final igual ou superior a 5 pontos

        \item Os alunos serão representados por uma estrutura que contém o nome do aluno e
            sua nota final

        \item Novamente será apresentado um código em C++, que utiliza laços

        \item O código em Haskell novamente substituirá o laço por recursão

        \item Atente que neste exemplo, e no anterior, a recursão é composta por um (ou mais)
            caso(s) base(s), e uma chamada recursiva
    \end{itemize}

\end{frame}

\begin{frame}[fragile]{Implementação da lista de aprovados em C++}
    \inputsnippet{c++}{5}{20}{codes/aprovados.cpp}
\end{frame}

\begin{frame}[fragile]{Implementação da lista de aprovados em Haskell}
    \inputcode{haskell}{codes/aprovados.hs}
\end{frame}

\begin{frame}[fragile]{Exemplo de uso de recursão: Produto Escalar}

    \begin{itemize}
        \item O terceiro exemplo de uso de recursão em substituição aos laços é o cálculo do
            produto escalar entre dois vetores

        \item Segundo a Álgebra Linear, dados dois vetores $\vec{u}, \vec{v}\in \mathbb{R}^n$,
            o produto escalar entre $\vec{u}$ e $\vec{v}$ é dado por
        \[
            \vec{u}\cdot \vec{v} = \sum_{i = 1}^n u_iv_i
        \]

        \item Observe que a própria definição sugere o uso de um laço, representado pelo 
            somatório

        \item Para utilizar a recursão, é preciso reinterpretar esta solução

        \item No caso base, se $\vec{u}, \vec{v}\in \mathbb{R}^0$, então $\vec{u}\cdot\vec{v} = 0$

        \item Se  $\vec{u}, \vec{v}\in\mathbb{R}^n$, então
        \[
            \vec{u}\cdot\vec{v} = u_1v_1 + \vec{r}\cdot\vec{s},
        \]
        onde $\vec{r} = (u_2, u_3, \ldots, u_n)$ e $\vec{s} = (v_2, v_3, \ldots, v_n)$
    \end{itemize}

\end{frame}

\begin{frame}[fragile]{Implementação do produto interno em C++}
    \inputsnippet{c++}{5}{13}{codes/dotproduct.cpp}
\end{frame}

\begin{frame}[fragile]{Implementação do produto interno em Haskell}
    \inputcode{haskell}{codes/dotproduct.hs}
\end{frame}

\begin{frame}[fragile]{Exemplo de uso de recursão: Verificação de primalidade}

    \begin{itemize}
        \item O último exemplo de uso de recursão para substituir laços é o teste de
            primalidade

        \item Dado um inteiro positivo $n$, a função \code{haskell}{is_prime(n)} deve
            retornar verdadeiro se $n$ é primo, e falso, caso contrário

        \item A complexidade da função que será apresentada é $O(\sqrt{n})$, pois se vale do
            fato de que, se $n$ é composto, ele tem ao menos um divisor próprio $d$ tal que
            $d \leq \sqrt{n}$

        \item Contudo, para evitar erros de precisão, a função \code{haskell}{sqrt} não é 
            utilizada explicitamente
    \end{itemize}

\end{frame}

\begin{frame}[fragile]{Implementação da verificação de primalidade em C++}
    \inputsnippet{c++}{5}{21}{codes/isprime.cpp}
\end{frame}

\begin{frame}[fragile]{Implementação da verificação de primalidade em Haskell}
    \inputcode{haskell}{codes/isprime.hs}
\end{frame}

\begin{frame}[fragile]{Mapas, filtros e reduções}

    \begin{itemize}
        \item Os \textbf{mapas}, os \textbf{filtros} e as \textbf{reduções} são funções de alta 
            ordem fundamentais em programação funcional

        \item Elas abstraem três conceitos fundamentais:
        \begin{enumerate}
            \item A partir de uma lista \code{haskell}{xs}, criar uma nova lista \code{haskell}{ys}
                tal que $y_i = f(x_i)$ para uma função $f$ dada (mapa)

            \item A partir de uma lista \code{haskell}{xs}, criar uma nova lista \code{haskell}{ys}
                formada pelos elementos \code{haskell}{x} de \code{haskell}{xs} que atendem a um
                predicado \code{haskell}{P} (filtro)

            \item Gerar um elemento \code{haskell}{y} a partir de uma lista \code{haskell}{xs}
                atráves da aplicação sucessiva de uma operação binária \code{haskell}{op} e um
                valor inicial \code{haskell}{x0} (redução)
        \end{enumerate}

        \item Todas as três técnicas recebem uma função como parâmetro

        \item A aplicação destas técnicas substituem, em vários casos, a necessidade dos laços das
            linguagens imperativas
    \end{itemize}

\end{frame}

\begin{frame}[fragile]{Mapas}

    \begin{itemize}
        \item Em Haskell, os mapas são implementados por meio da função \code{haskell}{map}
        
            \inputsyntax{haskell}{codes/map.hs}
        
        \item Um mapa recebe uma função \code{haskell}{f} que transforma um elemento do tipo 
            \code{haskell}{a} em um elemento do tipo \code{haskell}{b} e uma lista de elementos
            do tipo \code{haskell}{a}

        \item O retorno é uma lista de elementos do tipo \code{haskell}{b}, onde
            $b_i = f(a_i)$
        
        \item O uso de mapas simplifica o código e o torna mais legível
 
            \inputsyntax{haskell}{codes/capitalize_map.hs}
    \end{itemize}

\end{frame}

\begin{frame}[fragile]{Filtros}

    \begin{itemize}
        \item Em Haskell, os filtros são implementados por meio da função \code{haskell}{filter}:

            \inputsyntax{haskell}{codes/filter.hs}

        \item Um filtro recebe um predicato \code{haskell}{P} e uma lista de elementos do tipo
            \code{haskell}{[a]} e retorna uma nova lista do tipo \code{haskell}{[a]}

        \item Um elemento \code{haskell}{a} da lista de entrada estará na lista de saída
            se, e somente se, a expressão `\code{haskell}{P a}' for verdadeira

        \item A ordem relativa dos elementos é preservada

            \inputsyntax{haskell}{codes/hex.hs}
    \end{itemize}

\end{frame}

\begin{frame}[fragile]{Lista de aprovados usando filtros e mapas}
    \inputcode{haskell}{codes/aprovados_filter.hs}
\end{frame}

\begin{frame}[fragile]{Recursão de cauda}

    \begin{itemize}
        \item Uma função é dita \textbf{recursiva de cauda} (\textit{tail recursive}) se ela ou 
            retorna valores (casos-base) ou retorna chamadas de si mesma, com diferentes parâmetros

        \item Nem toda função recursiva é recursiva de cauda

        \item Por exemplo, a definição da função fatorial abaixo é recursiva, mas não recursiva
            de cauda:
        \[
            n! = \left\lbrace \begin{array}{ll} 1,& \mbox{se}\ n = 0\ \mbox{ou}\ n = 1 \\
                n\cdot (n - 1)!,& \mbox{caso contrário} \end{array}\right.
        \]

        \item Isto porque, na chamada recursiva, o retorno consiste no produto da chamada 
            de $(n - 1)!$ pelo parâmetro $n$ 

    \end{itemize}

\end{frame}

\begin{frame}[fragile]{Recursão de cauda}

    \begin{itemize}
        \item Contudo, esta definição pode ser modificada para que se torne recursiva de cauda:
        \[
            f(n, m) = \left\lbrace \begin{array}{ll} m,& \mbox{se}\ n = 0\ \mbox{ou}\ n = 1 \\
                f(n - 1, n\cdot m),& \mbox{caso contrário} \end{array}\right.
        \]

        \item Desde modo, $n! = f(n, 1)$

        \item Observe que a chamada recursiva agora consiste apenas em uma invocação da função
            $f$

        \item O parâmetro $m$ é denominado \textbf{acumulador}

        \item A recursão de cauda permite a \textbf{otimização de chamada de cauda}
            (\textit{tail call optimization -- TCO})

        \item Isto porque, neste caso, é possível evitar o uso da pilha de execução, reaproveitando
            um único registro de ativação a cada chamada
    \end{itemize}

\end{frame}

%\begin{frame}[fragile]{Reduções: {\it left fold}}

\section{Tipos de dados de usuário}

\begin{frame}[fragile]{Definição de novos tipos de dados}

    \begin{itemize}
        \item É possível introduzir novos tipos de dados por meio da palavra reservada
            \code{haskell}{data}

        \item A sintaxe é

            \inputsyntax{haskell}{codes/data.hs}

        \item Tanto o construtor de tipo quanto o construtor de valor devem iniciar em
            letra maiúscula

        \item Os $N$ tipos se referem aos tipos dos $N$ membros (campos) do novo tipo de dado

        \item O construtor de valor pode ser entendido como uma função qualquer

        \item Por exemplo, 

            \inputsyntax{haskell}{codes/student.hs}
    \end{itemize}

\end{frame}

\begin{frame}[fragile]{Definição de novos tipos de dados}

    \begin{itemize}
        \item O nome do  tipo e de seus valores são independentes

        \item Os nomes dos tipos são usados exclusivamente em suas definições

        \item Os construtores de valores são utilizados no programa para criar variáveis do
            tipo definido

        \item Quando não há ambiguidade, os nomes dos tipos e dos valores podem ser o mesmo

        \item Esta prática é normal e legal

        \item Haskell não permite a mistura de dois tipos de dados que são estruturalmente
            diferentes, mas tem nomes diferentes

        \item Por exemplo, no trecho abaixo,
            \inputsyntax{haskell}{codes/point2D.hs}
        \texttt{x} e \texttt{y} não são comparáveis, pois tem tipos distintos
    \end{itemize}

\end{frame}

\begin{frame}[fragile]{Sinônimos}

    \begin{itemize}
        \item Em Haskell é possível definir um \textbf{sinônimo} para um tipo já existe, com
            o intuito de ampliar o entendimento do código por meio do uso de nomes mais
            descritivos

        \item Isto pode ser feito por meio da palavra-reservada \code{haskell}{type}

        \item Por exemplo,

            \inputsyntax{haskell}{codes/student_type.hs}

        \item A definição de sinônimos é similar a feita em C/C++ por meio da palavra-reservada
            \code{cpp}{typedef}

    \end{itemize}

\end{frame}

\begin{frame}[fragile]{Tipos de dados algébricos}

    \begin{itemize}
        
        \item O tipo de dados \textbf{algébricos} podem ter mais de um construtor. Por exemplo:

            \inputsyntax{haskell}{codes/blood_groups.hs}

        \item Eles também não denominados tipos \textbf{enumeráveis}

        \item Cada um dos construtores pode ter zero ou mais parâmetros. Por exemplo:

            \inputsyntax{haskell}{codes/shape.hs}

        \item Com um único construtor, o tipo de dado algébrico equivale a uma \code{c}{struct}
            em C

        \item Com dois ou mais construtores, todos sem parâmetros, corresponde a uma 
            \code{c}{enum}

        \item Nos demais casos, pode ser visto como uma \code{haskell}{union} de C/C++
    \end{itemize}

\end{frame}

\begin{frame}[fragile]{\it Pattern Matching}

    \begin{itemize}
        \item Um padrão (\textit{pattern}) permite a extração de membros de um tipo

        \item Isto permite a definição de uma função por meio de vários padrões de entrada

        \item Quando a função for chamada, os valores dos parâmetros de entrada são 
            confrontados com os padrões definidos, um por vez, na ordem em que foram definidos

        \item Quando um casamento (\textit{matching}) for bem sucedido, a expressão associada
            definirá o valor de retorno da função

        \item Se nenhum casamento for bem sucedido, o resultado será um erro

        \item O símbolo `\code{haskell}{_}' (\textit{wildcard}) pode ser usado para casar qualquer
            valor

        \item Em alguns contextos, o \textit{pattern matching} é também denominado 
            \textbf{desconstrução}
    \end{itemize}

\end{frame}

\begin{frame}[fragile]{Exemplo de definição de função usando {\it pattern matching}}
    \inputcode{haskell}{codes/area.hs}
\end{frame}

\begin{frame}[fragile]{Exemplo de {\it pattern matching} em listas}
    \inputcode{haskell}{codes/sumlist.hs}
\end{frame}

\begin{frame}[fragile]{Exemplo de extração de membros usando {\it pattern matching}}
    \inputcode{haskell}{codes/student_members.hs}
\end{frame}

\begin{frame}[fragile]{O símbolo {\tt \$}} 

    \begin{itemize}
        \item O símbolo  \code{haskell}{$} é um operador em Haskell com um comportamento
            que não é óbvio a primeira vista

        \item O tipo do operador \code{haskell}{$} é

            \inputsyntax{haskell}{codes/dolar.hs}

        \item Ele recebe uma função \code{haskell}{f :: a -> b} e o parâmetro \code{haskell}{a},
            e retorna \code{haskell}{b = f(a)}

        \item Este é o mesmo comportamento do espaço em branco:

            \inputsyntax{haskell}{codes/space.hs}

        \item Assim, a primeira vista este operador parecer ser desnecessário, mas a sua
            importância vem de sua precedência

            \inputsyntax{haskell}{codes/precedence.hs}
    \end{itemize}

\end{frame}

\begin{frame}[fragile]{O símbolo {\tt \$}} 

    \begin{itemize}
        \item \code{haskell}{infixr} nos diz que este é um operador associativo à direita

        \item O valor \code{haskell}{0} (zero) corresponde a menor precedência possível

        \item Já a aplicação de função (operador espaço) é associativo à esquerda e tem a
            maior precedência possível

        \item Assim, ele serve para mudar a associatividade e precedência da aplicação de funções

        \item Por exemplo, a expressão

            \inputsyntax{haskell}{codes/invalid.hs}

        é inválida, pois a associatividade e a precedência da aplicação faz com que ela seja
         interpretada como

            \inputsyntax{haskell}{codes/invalid2.hs}

        \item Já com o uso do operador dólar temos

            \inputsyntax{haskell}{codes/valid.hs}
    \end{itemize}

\end{frame}

\begin{frame}[fragile]{Notação de registro}

    \begin{itemize}
        \item Escrever funções de acesso para os membros de um tipo de dado é tedioso e
            propenso a erros

        \item Haskell oferece uma definição alternativa, que permite listar as funções de
            acesso no momento da definição do novo tipo de dado

        \item Por exemplo, o tipo \code{haskell}{Student} definido anteriormente poderia
            ser definido da seguinte forma:

            \inputsyntax{haskell}{codes/student_record.hs}

        \item Esta notação também pode ser utilizada para criar variáveis deste novo tipo

        \item A impressão padrão herdada de \code{haskell}{Show} também fica mais elaborada
    \end{itemize}

\end{frame}

\begin{frame}[fragile]{Exemplo de criação de variável com notação de registro}
    \inputcode{haskell}{codes/student_record_variables.hs}
\end{frame}

\begin{frame}[fragile]{Tipos parametrizáveis}

    \begin{itemize}
        \item Em Haskell é possível criar tipos de dados parametrizados, por meio da introdução
            de uma variável na definição de tipo

        \item Na biblioteca padrão (\code{haskell}{Prelude}) é definido o tipo parametrizado
            \code{haskell}{Maybe}:

            \inputsyntax{haskell}{codes/maybe.hs}

        \item A variável \code{haskell}{a} nesta definição significa um tipo de dado qualquer

        \item Este tipo de dado é utilizado para retornos de funções que podem estar ausentes

        \item Por exemplo, a função \code{haskell}{head} retorna um erro quando aplicada à
            lista vazia

            \inputsyntax{haskell}{codes/head_error.hs}

        \item Uma versão ``segura'' de \code{haskell}{head} pode ser implementada da seguinte
            maneira:

            \inputsyntax{haskell}{codes/safehead.hs}
    \end{itemize}

\end{frame}

\begin{frame}[fragile]{Tipos recursivos}

    \begin{itemize}
        \item Um tipo é dito \textbf{recursivo} se ele for definido em termos de si próprio

        \item Por exemplo, uma lista parametrizada pode ser definida como

            \inputsyntax{haskell}{codes/list.hs}

        \item Os tipos recursivos devem ter ao menos um caso base (construtor que não se refere
            ao próprio tipo)

        \item Assim, eles devem ter, no mínimo, dois construtores distintos

        \item Uma árvore binária pode ser definida como

            \inputsyntax{haskell}{codes/bst.hs}
    \end{itemize}

\end{frame}

\begin{frame}[fragile]{Exemplo de definição e uso de tipo recursivo}
    \inputcode{haskell}{codes/list_example.hs}
\end{frame}

\begin{frame}[fragile]{Variáveis locais}

    \begin{itemize}
        \item Em Haskell há duas formas de introduzir variáveis locais a uma função

        \item A primeira delas é utilizando uma expressão \code{haskell}{let}, cuja sintaxe é

            \inputsyntax{haskell}{codes/let.hs}

        \item Cada linha define uma nova variável

        \item O escopo destas variáveis é a expressão que define o valor da função, e também
            as linhas de definição de variáveis subsequentes

        \item Se o nome de uma das variáveis locais coincidir com o nome de um dos parâmetros da
            função, prevalecerá a variável local (\textit{shadowing}), o que pode levar a 
            expressões confusas ou \textit{bugs}

    \end{itemize}

\end{frame}

\begin{frame}[fragile]{Variáveis locais}

    \begin{itemize}
        \item A segunda forma de se declarar variáveis locais é por meio de uma expressão
            \code{haskell}{where}, cuja sintaxe é

            \inputsyntax{haskell}{codes/where.hs}

        \item A diferença entre as expressões \code{haskell}{where} e \code{haskell}{let} é que
            a primeira prioriza a expressão, deixando os detalhes para depois

        \item O escopo das variáveis definidas na expressão \code{haskell}{where} é a expressão
            da função, que precede o bloco, e o próprio bloco \code{haskell}{where}

        \item Além de variáveis locais, é possível definir funções locais nas expressões
            \code{haskell}{let} e \code{haskell}{where}

        \item A escolha entre estas duas alternativas é uma questão de estilo
    \end{itemize}

\end{frame}

\begin{frame}[fragile]{Exemplo de uso de variáveis locais}
    \inputcode{haskell}{codes/polynomial.hs}
\end{frame}

\begin{frame}[fragile]{Indentação}

    \begin{itemize}
        \item Em Haskell, a indentação e espaços em branco são utilizados para delimitar os
            blocos

        \item A primeira expressão de um programa pode começar em qualquer coluna do texto, desde
            que as demais expressões de mesmo nível comecem nesta mesma coluna

        \item Se a expressão é seguida de uma linha em branco ou por uma expressão em uma coluna
            mais à direita, ela é considerada uma continuação da expressão anterior

        \item Após um \code{haskell}{let} ou um \code{haskell}{where}, o compilador memorizará
            a posição do próximo \textit{token}: expressões nas próximas linhas que comecem na
            mesma posição serão consideradas novas entradas destes blocos

        \item Por conta destas regras, é indicado o uso de espaços, e não de tabulações, na
            indentação do código

        \item Embora não seja comum, os blocos podem ser delimitados por chaves (`\texttt{\{}' e
            `\texttt{\}}'): neste caso, as regras acima podem ser violadas, usando-se 
            ponto-e-vírgula para separar as expressões
            
    \end{itemize}

\end{frame}

\begin{frame}[fragile]{Expressões {\it case}}

    \begin{itemize}
        \item As expressões \code{haskell}{case} permitem confrontar uma expressão com vários
            padrões

        \item A sintaxe é

            \inputsyntax{haskell}{codes/case.hs}

        \item O valor da expressão é confrontado com cada um dos padrões, na ordem descrita

        \item Caso um casamento seja válido, o valor da expressão \code{haskell}{case} será o
            valor indicado após a seta (\code{haskell}{->})

        \item Também é possível usar o \textit{wildcard}
    \end{itemize}

\end{frame}

\begin{frame}[fragile]{Exemplo de expressão {\it case}}
    \inputcode{haskell}{codes/frommaybe.hs}
\end{frame}

\begin{frame}[fragile]{Guardas}

    \begin{itemize}
        \item O uso de \textit{pattern matching} é feito apenas com valores dos objetos

        \item Para testes mais elaborados, com condicionais, Haskell disponibiliza as 
            \textbf{guardas}

        \item Cada padrão a ser casado pode ser seguido por uma ou mais guardas, sendo elas
            expressões do tipo \code{haskell}{Bool}

        \item Cada guarda é introduzida pelo símbolo \code{haskell}{|}, e seguida de um
            \code{haskell}{=} (ou \code{haskell}{->}, no caso de expressões \code{haskell}{case})

        \item Após o símbolo, segue a expressão a ser utilizada caso o padrão seja casado e a
            guarda verdadeira

        \item A expressão \code{haskell}{otherwise} é atada ao valor \code{haskell}{True}, de modo
            que é uma guarda sempre verdadeira, o que melhora a legibilidade do código
    \end{itemize}

\end{frame}

\begin{frame}[fragile]{Exemplo de uso de guardas}
    \inputcode{haskell}{codes/guards.hs}
\end{frame}

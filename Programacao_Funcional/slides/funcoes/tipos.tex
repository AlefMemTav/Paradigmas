\section{Tipos de dados de usuário}

\begin{frame}[fragile]{Definição de novos tipos de dados}

    \begin{itemize}
        \item É possível introduzir novos tipos de dados por meio da palavra reservada
            \code{haskell}{data}

        \item A sintaxe é

            \inputsyntax{haskell}{codes/data.hs}

        \item Tanto o construtor de tipo quanto o construtor de valor devem iniciar em
            letra maiúscula

        \item Os $N$ tipos se referem aos tipos dos $N$ membros (campos) do novo tipo de dado

        \item O construtor de valor pode ser entendido como uma função qualquer

        \item Por exemplo, 

            \inputsyntax{haskell}{codes/student.hs}
    \end{itemize}

\end{frame}

\begin{frame}[fragile]{Definição de novos tipos de dados}

    \begin{itemize}
        \item O nome do  tipo e de seus valores são independentes

        \item Os nomes dos tipos são usados exclusivamente em suas definições

        \item Os construtores de valores são utilizados no programa para criar variáveis do
            tipo definido

        \item Quando não há ambiguidade, os nomes dos tipos e dos valores podem ser o mesmo

        \item Esta prática é normal e legal

        \item Haskell não permite a mistura de dois tipos de dados que são estruturalmente
            diferentes, mas tem nomes diferentes

        \item Por exemplo, no trecho abaixo,
            \inputsyntax{haskell}{codes/point2D.hs}
        \texttt{x} e \texttt{y} não são comparáveis, pois tem tipos distintos
    \end{itemize}

\end{frame}

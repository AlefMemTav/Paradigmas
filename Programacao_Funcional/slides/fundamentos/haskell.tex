\section{Haskell}

\begin{frame}[fragile]{Haskell}

    \begin{itemize}
        \item Haskell é uma linguagem de programação moderna, puramente funcional

        \item Ela implementa valoração não-estrita, polimorfismo de tipos, funções de alta-ordem
            e um sistema de tipagem forte e estrito

        \item Em termos numéricos, ela oferece inteiros de precisão arbitrária, números racionais,
            e números em ponto flutuante e variáveis booleanas

        \item Ela foi desenvolvida por um comitê, cuja primeira versão data de 1990

        \item Dentre as principais motivações para a criação do Haskell estavam:

        \begin{enumerate}[i.]
            \item unificar os esforços de dezenas de diferentes linguagens funcionais,
            \item ter uma linguagem funcional simples e apropriada para ensino, e
            \item criar uma linguagem funcional livre
        \end{enumerate}
    \end{itemize}

\end{frame}

\begin{frame}[fragile]{GHC}

    \begin{itemize}
        \item GHC (\textit{Glasgow Haskell Compiler}) é um compilador (\texttt{ghc}),
            interpretador (\texttt{runghc}) e um ambiente interativo (\texttt{ghci}), de 
            código aberto, para a linguagem Haskell

        \item Ele tem bom suporte para paralelismo, e gera códigos rápidos, em especial em
            programação concorrentes

        \item O GHC oferece, por padrão, uma variedade de bibliotecas, e outras podem ser 
            encontradas em \href{https://hackage.haskell.org/}{Hackage}
 
        \item Há versões para plataformas Windows, Linux e MacOS, dentre outras

        \item Em Linux, ele pode ser instalado com o comando

            \inputsyntax{bash}{codes/install.sh}

        \item Para mudar o \textit{prompt} do GHCi, use o comando \code{haskell}{:set}

            \inputsyntax{haskell}{codes/ghci.hs}

    \end{itemize}

\end{frame}

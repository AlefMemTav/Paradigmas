\section{Programação Orientada a Objetos}

\begin{frame}[fragile]{Introdução}

    \begin{itemize}
        \item A \textbf{programação orientada a objetos} é um paradigma de programação baseada em 
            objetos

        \item \textbf{Objetos} são estruturas de dados que armazenam informações 
            (\textbf{atributos}) e procedimentos (\textbf{métodos})

        \item Os programas consistem em interações entre os diferentes objetos 

        \item Estas interações são conduzidas por meio de trocas de \textbf{mensagens}

        \item Os objetos são instâncias de \textbf{classes}, as quais determinam os tipos, atributos
            e métodos dos objetos

        \item Muitas linguagens multiparadigmas suportam a orientação a objetos

        \item Algumas outras, como Smalltalk, suporta apenas o paradigma da programação orientada a
            objetos
    \end{itemize}

\end{frame}

\begin{frame}[fragile]{Histórico}

    \begin{itemize}
        \item O termo objeto, no contexto de programação orientada a objetos, surgiu no MIT no
            final dos anos 1950 e no início dos anos 1960

        \item Ele se referia a um item identificado com atributos

        \item Também no MIT, Iva Sutherland desenvolveu um software denominado Sketchpad, ancestral
            dos programas CAD, e definiu os termos objeto e instância

        \item A linguagem Simula67 introduziu o conceito formal de objetos em programação, e tamem
            a noção de classes

        \item A expressão ``programação orientada a objetos'' foi introduzida pela linguagem 
            SmallTalk, nos anos 1970

        \item Esta linguagem faz uso de objetos e mensagens como base de suas computações, e as
            classes podiam ser modificadas dinamicamente

        \item Nos anos 1990 este paradigma se tornou a metodologia dominante de desenvolvimento 
            de software, obtendo suporte na maior parte das linguagens então existentes
    \end{itemize}

\end{frame}

\begin{frame}[fragile]{Visão geral}

    \begin{itemize}
        \item Além de unir dados e funções, os objetos também são \textbf{tipos de dados 
            abstratos}, isto é, são definidos por suas interfaces, e não por suas implementações

        \item Os objetos também oferecem suporte para \textbf{polimorfismo} e para \textbf{herança}

        \item Os objetos são organizados em hierarquias de \textbf{classes}

        \item As classes e as \textbf{subclasses} são análogos aos conjuntos e subconjuntos da
            matemática

        \item Os objetos são \textbf{modulares}, no sentido que são responsáveis por seus dados
            e pelo próprios comportamentos

        \item Esta característica é denominada \textbf{encapsulamento}
           
    \end{itemize}

\end{frame}

\begin{frame}[fragile]{Visão geral}

    \begin{itemize}
        \item Na programação orientada a objetos os dados devem, sempre que possível, ser protegidos
            de acessos diretos

        \item O acesso aos dados do objetos deve ser feito por meio dos métodos de sua
            \textbf{interface}
 
        \item Outro aspecto importante deste paradigma é o \textbf{reuso} de código

        \item São duas as principais formas de reuso: construir novos objetos através da
            \textbf{composição} de objetos já existentes ou estender um objeto por meio de 
            herança

        \item Um objeto (ou módulo) está \textbf{aberto} se ele permite sua extensão por herança

        \item Ele está \textbf{fechado} se sua interface é estável e bem definida, de modo que
            não pode (ou deve) ser estendida
    \end{itemize}

\end{frame}

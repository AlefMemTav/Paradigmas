\section{\it GNU Smalltalk}

\begin{frame}[fragile]{GNU Smalltalk}

    \begin{itemize}
        \item GNU Smalltalk é uma implementação livre da linguagem Smalltalk-80

        \item Esta está disponível na maioria das distribuições Linux e outras plataformas
            que suportam o padrão POSIX

        \item Ela utiliza a licença GNU GPL v2 em algumas partes e a GNU Lesser GPL v2 em outras

        \item Ela pode ser instalada através do comando

            \inputsyntax{bash}{codes/smalltalk.sh}

        \item Para validar a instalação, pode-se utilizar o comando

            \inputsyntax{bash}{codes/version.sh}

        \item O ambiente interativo pode ser iniciado com o comando

            \inputsyntax{bash}{codes/run.sh}
    \end{itemize}

\end{frame}

\begin{frame}[fragile]{\it Hello World!}

    \begin{itemize}
        \item Para imprimir a mensagem `\textit{Hello World!}' no ambiente interativo basta
            invocar o método \code{smalltalk}{printNl}

            \inputsyntax{smalltalk}{codes/helloworld.st}

        \item A mensagem é impressa duas vezes: a primeira por conta do método 
            \code{smalltalk}{printNl}, a segunda por conta do ambiente interativo, que imprime
            o valor resultante da expressão inserida

        \item Observe que o texto antecede o método

        \item Isto por que, em Smalltalk, a sintaxe para passar uma mensagem para um objeto é

            \inputsyntax{smalltalk}{codes/syntax.st}

        \item Note a ausência de parêntesis e da notação que utiliza o ponto final, comum em
            muitas outras linguagens que suportam a programação orientada à objetos
    \end{itemize}

\end{frame}

\begin{frame}[fragile]{Operadores lógicos e aritméticos}

    \begin{itemize}
        \item Em Smalltalk, todos os tipos são objetos

        \item O código abaixo ilustra o polimorfismo do método \code{smalltalk}{printNl}

            \inputsyntax{smalltalk}{codes/printint.st}

        \item Por conta do comportamento do ambiente interativo, o método \code{smalltalk}{printNl}
            será omitido dos demais código

        \item A linguagem tem suporte para os operadores lógicos

            \inputsyntax{smalltalk}{codes/logical.st}

        \item E também para os operadores aritméticos

            \inputsyntax{smalltalk}{codes/arithmetic.st}
    \end{itemize}

\end{frame}

\begin{frame}[fragile]{Vetores}

    \begin{itemize}
        \item Em Smalltalk a sintaxe para a declaração de vetores é

            \inputsyntax{smalltalk}{codes/array.st}
        onde $N$ é a quantidade de elementos

        \item Inicialmente, todos os elementos estão vazios (\code{smalltalk}{nil})

        \item O tamanho do vetor é fixo, e não pode ser alterado

        \item A mensagem \code{smalltalk}{at: i} permite o acesso ao elemento que ocupa a
            $i$-ésima posição

            \inputsyntax{smalltalk}{codes/at.st}

        \item Para modificar o valor do elemento que ocupa a $i$-ésima posição, é preciso
            combinar as mensagens \code{smalltalk}{at: i} e \code{smalltalk}{put: value}

            \inputsyntax{smalltalk}{codes/put.st}
    \end{itemize}

\end{frame}

\begin{frame}[fragile]{Vetores}

    \begin{itemize}
        \item Ao contrário de C e outras linguagens, Smalltalk segue o padrão da matemática
            e atribui a posição 1 ao primeiro elemento do vetor

        \item Os elementos do vetor pode ser de quaisquer tipos

            \inputsyntax{smalltalk}{codes/fill.st}
        
        \item Em Smalltalk, os comentários são delimitados por aspas duplas (\texttt{"})

        \item A tentativa de acessso a um índice fora do intervalo $[1, N]$ resulta em um
            erro do tipo `\textit{index out of range}'

        \item Um vetor também pode ser construído com a sintaxe

            \inputsyntax{smalltalk}{codes/const_array.st}

        \item A mensagem \code{smalltalk}{size} retorna o número de elementos do vetor

            \inputsyntax{smalltalk}{codes/size.st}
    \end{itemize}

\end{frame}

\begin{frame}[fragile]{Conjuntos}

    \begin{itemize}
        \item Em Smalltalk o conjunto (\code{smalltalk}{set}) representa uma coleção de elementos
            únicos

        \item A sintaxe para a declaração de um conjunto é

            \inputsyntax{smalltalk}{codes/set.st}

        \item A mensagem \code{smalltalk}{add: value} adiciona o valor indicado ao conjunto

        \item Várias expressões podem ser concatenadas para serem executadas simultaneamente
            por meio do operador ponto final (\texttt{.})

            \inputsyntax{smalltalk}{codes/add.st}

        \item Se as várias mensagem tem o mesmo objeto como destinatário, a notação
            de ponto pode ser simplificada por meio do uso do operador ponto-e-vírgula
            (\texttt{;})

            \inputsyntax{smalltalk}{codes/semicolon.st}
    \end{itemize}

\end{frame}

\begin{frame}[fragile]{Conjuntos}

    \begin{itemize}

        \item Para testar se o elemento $y$ pertence ao conjunto, basta usar a mensagem
            \code{smalltalk}{includes: y}

            \inputsyntax{smalltalk}{codes/includes.st}

        \item A mensagem \code{smalltalk}{remove: y} permite remover um elemento do
            conjunto

            \inputsyntax{smalltalk}{codes/remove.st}

        \item Se $y$ não pertence ao conjunto, será emitido um erro do tipo `\textit{object
            not found}'
    \end{itemize}

\end{frame}

\begin{frame}[fragile]{Dicionários}

    \begin{itemize}
        \item Os dicionários permitem armazenar elementos (valores) indexados por
            chaves

        \item A sintaxe para a declaração de um dicionário é

            \inputsyntax{smalltalk}{codes/dict.st}

        \item Tanto as chaves quanto os valores podem ser de tipos arbitrários

            \inputsyntax{smalltalk}{codes/key_values.st}

        \item A mensagem \code{smalltalk}{includes: y} verifica se $y$ está entre os
            valores do dicionário, enquanto que a mensagem \code{smalltalk}{includesKey: z}
            procura $z$ entre suas chaves

            \inputsyntax{smalltalk}{codes/includeskey.st}
    \end{itemize}

\end{frame}


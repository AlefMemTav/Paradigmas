\section{Estruturas de repetição}

\begin{frame}[fragile]{Laços}

    \begin{itemize}
        \item Smalltalk possui métodos que permitem executar um bloco repetidas vezes

        \item Os blocos possuem o método \code{smalltalk}{whileTrue}, que recebe um bloco de
            comandos que será executado enquanto o bloco que o invocou seja avaliado como
            verdadeiro

            \inputsyntax{smalltalk}{codes/whiletrue.st}

        \item O bloco de comandos deve, em algum momento, modificar as variáveis que compõem a
            condição, caso contrário o laço executará indefinidamente

        \item O método \code{smalltalk}{whileFalse} tem comportamento semelhante, executando o
            bloco de comandos enquanto a condição for falsa

            \inputsyntax{smalltalk}{codes/whilefalse.st}

    \end{itemize}

\end{frame}

\begin{frame}[fragile]{Exemplo de uso de laços em SmallTalk}
    \inputsnippet{smalltalk}{1}{21}{codes/num_digits.st}
\end{frame}

\begin{frame}[fragile]{Repetição}

    \begin{itemize}
        \item Se um bloco de comandos deve ser executado exatamente $n$ vezes, uma forma mais
            concisa e apropriada do que  \code{smalltalk}{whileTrue} é o método 
            \code{smalltalk}{timesRepeat} dos inteiros

            \inputsyntax{smalltalk}{codes/timesrepeat.st}

        \item De fato, são equivalentes os códigos

            \inputsyntax{smalltalk}{codes/while_n_repeat.st}

        e
            \inputsyntax{smalltalk}{codes/times_n_repeat.st}

        \item Os blocos também podem, opcionalmente, ter variáveis locais ou, no caso de
            coleções, capturar argumentos 

        \item A sintaxe de um bloco é

            \inputsyntax{smalltalk}{codes/blocks.st}
    \end{itemize}

\end{frame}

\begin{frame}[fragile]{Exemplo de uso de blocos repetidos em SmallTalk}
    \inputsnippet{smalltalk}{1}{21}{codes/fib.st}
\end{frame}

\begin{frame}[fragile]{Travessia em contêineres}

    \begin{itemize}
        \item Em SmallTalk os contêineres oferecem o método \code{smalltalk}{do}, o qual 
            recebe um bloco de comandos como argumento e o executa para cada um dos elementos
            do contêiner

        \item Os valores destes elementos serão passados como argumentos dos blocos

        \item O contêiner \code{smalltalk}{Interval}, que abstrai uma sequência de inteiros,
            pode ser utilizado para simular o laço \texttt{for} das linguagens imperativas:

            \inputsyntax{smalltalk}{codes/interval_do.st}

        \item Os valores $a$ e $b$ representam os valores iniciais e finais da variável $i$

        \item A partir de $a$, os valores serão incrementados em $step$ a cada iteração

        \item Se o método \code{smalltalk}{by} não for utilizado, $step$ terá valor igual a 1
    \end{itemize}

\end{frame}

\begin{frame}[fragile]{Exemplo de uso de travessia de contêineres}
    \inputsnippet{smalltalk}{1}{21}{codes/primes.st}
\end{frame}

\section{Definição do cálculo $\lambda$}

\begin{frame}[fragile]{Cálculo $\lambda$}

    \begin{block}{Termos-$\lambda$}
        O conjunto $\Lambda$ dos termos-$\lambda$ (ou expressões-$\lambda$, ou 
        simplesmente lambdas) é definido por
        meio de um conjunto de variáveis $V$ através das regras de aplicação e abstração,
        dadas a seguir:
        \begin{enumerate}
            \item $x\in V \Rightarrow x\in \Lambda$ (expressão)
            \item $M, N\in V \Rightarrow MN\in \Lambda$ (aplicação)
            \item $M\in \Lambda, x\in V \Rightarrow \lambda x.M$ (abstração)
        \end{enumerate}
    \end{block}

    \vspace{0.1in}

    \textbf{Observação}: informalmente, a aplicação equivale ao cálculo da função 
        $M$ com argumento $N$, isto é $M(N)$; a abstração corresponde a definição da função 
        $f(x) = M$.

\end{frame}

\begin{frame}[fragile]{Exemplos de termos-$\lambda$}

    \begin{enumerate}
        \item O termo-$\lambda$ mais simples possível é composto por uma única variável (por 
            exemplo, $x$)

        \item A função identidade $\lambda x.x$ é um exemplo de abstração

        \item Parêntesis podem ser utilizados para clarificar uma expressão, ou para remover
            ambiguidades

        \item O termo $(\lambda x.x)y$ é a aplicação da função identidade ao termo $y$

        \item A aplicação é associativa à esquerda:
        \[
            M_1M_2\ldots M_N = (((M_1M_2)M_3)\ldots M_N)
        \]

        \item O termo $\lambda y.(\lambda x.M)$ equivale a uma função de duas variáveis

        \item Uma notação alternativa para o termo anterior é 
        \[
            \lambda yx.M = \lambda y.(\lambda x.M)
        \]

        \item A abstração é associativa à direita:
        \[
            \lambda x_1x_2\ldots x_N.M = \lambda x_1.(\lambda x_2.(\ldots \lambda x_N.M))
        \]
    \end{enumerate}

\end{frame}

\begin{frame}[fragile]{Variáveis livres e atadas ({\it bound})}

    \begin{itemize}
        \item A abstração $\lambda x.M$ une (ata, \textit{to bind}) a variável livre $x$ ao
            termo (expressão) lambda $M$

        \item Uma variável não precedida por um símbolo $\lambda$ que a une a uma expressão
            é denominada variável \textbf{livre}

        \item Na expressão 
        \[
            \lambda x.xy
        \]
        a variável $x$ é atada e a variável $y$ é livre

        \item Uma mesma variável pode ser livre e atada em uma mesma expressão. Por exemplo,
            na expressão
        \[
            (\lambda x.xy)(\lambda y.y)
        \]
        a variável $y$ é livre no termo entre parêntesis à esquerda, e atada no termo da direita

    \end{itemize}

\end{frame}

\begin{frame}[fragile]{Substituições}

    \begin{block}{Substituição}
    A \textbf{substituição} de todas as ocorrências da variável livre $x$ por $N$ em $M$, cuja
    notação é $M[x:=N]$, é definida por
    \begin{enumerate}[i.]
        \item $x[x:=N] \equiv N$
        \item $y[x:=N] \equiv y$, se $y\not\equiv x$
        \item $(M_1M_2)[x:=N] \equiv (M_1[x:=N])(M_2[x:=N])$
        \item $(\lambda y.M_1)[x:=N] \equiv \lambda y.(M_1[x:=N])$
    \end{enumerate}
    \end{block}

\end{frame}

\begin{frame}[fragile]{Exemplos de substituição}

    \begin{enumerate}
        \item Exemplo de substituição pela regra 3:
        \[
            ((\lambda x.xyz)(\lambda y.xzy))[z:=N] \equiv (\lambda x.xyN)(\lambda y.xNy) 
        \]

        \item Exemplo de substituição pela regra 4:
        \[
            (\lambda x.xy)[y:=N] \equiv \lambda x.xN
        \]

        \item Exemplo de substituição pelas regras 2 e 4:
        \[
            (\lambda x.xy)[z:=N] \equiv \lambda x.xy
        \]

        \item $(\lambda x.xy)[x:=N]$ não é uma expressão lambda válida, pois as substituições
            devem ser feitas em termos de variáveis livres, e $x$ é atada na expressão entre
            parêntesis
    \end{enumerate}

\end{frame}

\begin{frame}[fragile]{Operações de Redução}

    \begin{block}{Axiomas de Redução}
        \begin{enumerate}
            \item A conversão-$\alpha$ permite a troca das variáveis atadas de uma expressão:
            \[
                \lambda x.M \equiv \lambda y.(M[x:=y])
            \]

            \item A redução-$\beta$ associa a aplicação com a substituição:
            \[
                (\lambda x.M)N \equiv M[x:=N]
            \]
        \end{enumerate}
    \end{block}

    \vspace{0.1in}

    \textbf{Observação}: a conversão-$\alpha$ é utilizada, primariamente, para evitar colisões
        de nomes; a redução-$\beta$ é o axioma fundamental do cálculo $\lambda$.
\end{frame}

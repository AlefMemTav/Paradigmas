\section{Valores lógicos}

\begin{frame}[fragile]{Contextualização}

    \begin{itemize}
        \item Sendo originalmente um sistema lógico, o cálculo $\lambda$ possui apenas
            dois termos primitivos: a letra grega lambda ($\lambda$) e o ponto final ($.$)

        \item Os axiomas de construção de termos-$\lambda$ (expressões, aplicação e abstração)
            permitem a definição de novos termos a partir destes dois termos primitivos

        \item Deste modo, os valores lógicos da Lógica Proposicional Booleana (verdadeiro e falso)
            devem ser igualmente definidos como termos-$\lambda$

        \item As operações lógicas, que permitem a construção de proposições compostas, também
            devem ser definidas como termos-$\lambda$
    \end{itemize}

\end{frame}

\begin{frame}[fragile]{Valores lógicos}

    \begin{block}{Verdadeiro e Falso}
        O valor lógico \textbf{verdadeiro} pode ser representado pela expressão-$\lambda$
        \[
            T \equiv \lambda xy.x
        \] e o valor lógico \textbf{falso} pode ser representado por
        \[
            F \equiv \lambda xy.y
        \]
    \end{block}

    \vspace{0.1in}

    \textbf{Observação}: veja que $T$ é, de fato, o combinador $\mathbf{K}$, e que $F$ é o 
    combinador $\mathbf{K_*}$, o qual é extensionalmente igual ao combinador $\mathbf{SK}$, pois
    \[
        \mathbf{SK}xy \equiv \mathbf{K}y(xy) \equiv y \equiv Fxy
    \]
\end{frame}

\begin{frame}[fragile]{Estrutura \texttt{if-then-else}}

    \begin{block}{\texttt{if-then-else}}
        Se $p$ é igual a $T$ ou a $F$, a expressão-$\lambda$
        \[
            I_F \equiv \lambda pab.pab
        \]
        é corresponde ao construto \texttt{if-then-else}. 
    \end{block}

    \vspace{0.1in}

    \textbf{Observação}: para visualizar esta correspondência, observe que
    \[
        (I_F) Tab \equiv Tab \equiv a
    \]
    e que
    \[
        (I_F) Fab \equiv Fab \equiv b
    \]
\end{frame}

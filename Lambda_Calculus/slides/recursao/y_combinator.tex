\section{Combinador $\mathbf{Y}$}

\begin{frame}[fragile]{Teorema do Ponto Fixo}

    \begin{block}{Teorema do Ponto Fixo}
        Para qualquer termo-$\lambda$ $G$ existe um termo $X$ tal que $GX\equiv X$.
    \end{block}

    \vspace{0.2in}

    \begin{block}{Demonstração}
        Seja $G$ um termo-$\lambda$ qualquer. Defina $W\equiv (\lambda x.G(xx))$ e $X=WW$. 

        Deste modo,
        \[
            X\equiv WW \equiv (\lambda x.G(xx))W \equiv G(WW)\equiv GX
        \]
    \end{block}
\end{frame}

\begin{frame}[fragile]{Combinador $\mathbf{Y}$}

    \begin{block}{Proposição (Combinador $\mathbf{Y}$)}
    O \textbf{combinador} $\mathbf{Y}$
    \[
        \mathbf{Y}\equiv \lambda f.(\lambda x.f(xx))(\lambda x.f(xx))
    \]
    é um termo-$\lambda$ tal que, para qualquer termo $G$,
    \[
        \mathbf{Y}G\equiv G(\mathbf{Y}G)
    \]
    \end{block}

    \vspace{0.1in}

    \begin{block}{Demonstração}
        Seja $G$ um termo-$\lambda$ qualquer. Daí
        \begin{align*}
            \mathbf{Y}G &\equiv (\lambda f.(\lambda x.f(xx))(\lambda x.f(xx)))G \\
             &\equiv (\lambda x.G(xx))(\lambda x.G(xx)) \\
             &\equiv G((\lambda x.G(xx))(\lambda x.G(xx))) \\
             &\equiv G(\lambda f.(\lambda x.f(xx))(\lambda x.f(xx)))G) \equiv G(\mathbf{Y}G)
        \end{align*}
    \end{block}
\end{frame}

\begin{frame}[fragile]{Observações sobre o combinador $\mathbf{Y}$}

    \begin{itemize}
        \item Veja que, para qualquer termo-$\lambda$ $G$, $\mathbf{Y}G$ é um ponto fixo de $G$

        \item Esta propriedade é o que faltava para a definição completa da recursão, pois
            ao aplicar $(\mathbf{Y}G)$ ao parâmetro $x$ da recursão, o resultado é 
        \[
            (\mathbf{Y}G)x \equiv G(\mathbf{Y}G)x,
        \]
        ou seja, o termo $G$ é aplicado aos parâmetros $\mathbf{Y}G$ e $x$, o que permite invocar
        $G$ novamente quantas vezes forem necessárias

        \item Assim, para definir uma função recursiva $\mathbf{Y}\Gamma$ no cálculo-$\lambda$, basta 
        determinar o predicado $P$ e as funções $g$ e $h$ que compõem a função $\Gamma$
    \end{itemize}

\end{frame}

\begin{frame}[fragile]{Exemplo de recursão: fatorial}

    \begin{huge}
    \[
        !n =  \left\lbrace \begin{array}{ll}
                    1, & \mbox{se}\ n = 0, \\
                    n \times !(n - 1), & \mbox{caso contrário}
                \end{array} \right.
    \]
    \end{huge}

    \begin{itemize}
        \item A notação está ``invertida'' para ficar consistem com a notação prefixada
            do cálculo lambda

        \item Na notação de recursão do cálculo lambda, $P\equiv Z, g \equiv 1$ e 
            $h\equiv \times x(f(Px))$, onde $\times ab$ é a multiplicação dos 
            naturais $a$ e $b$ e $Pn$ é o antecessor do natural $n$. 

        \item Deste modo, $! = \mathbf{Y} \Gamma$, onde
        \[
            \Gamma\equiv \lambda fx.(Zx)1(\times x(f(Px)))
        \]
    \end{itemize}
\end{frame}

\begin{frame}[fragile]{Exemplo de aplicação do fatorial}

    \begin{align*}
        !3 \equiv (\mathbf{Y}\Gamma )3 &\equiv \Gamma (\mathbf{Y}\Gamma )3 \\
        &\equiv (\lambda fx.(Zx)1(\times n(f(Pn))))(\mathbf{Y}\Gamma )3 \\
        &\equiv (Z3)1(\times 3((\mathbf{Y}\Gamma )(P3))) \\
        &\equiv F1(\times 3((\mathbf{Y}\Gamma )(P3))) \\
        &\equiv \times 3((\mathbf{Y}\Gamma )2)) \equiv \times 3(\Gamma (\mathbf{Y}\Gamma )2)) \\
        &\equiv \times 3((Z2)1(\times 2((\mathbf{Y}\Gamma )(P2)))) \\
        &\equiv \times 3(\times 2((\mathbf{Y}\Gamma )1)) \equiv \times 3(\times 2(\Gamma (\mathbf{Y}\Gamma )1)) \\
        &\equiv \times 3(\times 2((Z1)1(\times 1((\mathbf{Y}\Gamma )(P1))) \\
        &\equiv \times 3(\times 2(\times 1((\mathbf{Y}\Gamma )0))) \\
        &\equiv \times 3(\times 2(\times 1((Z0)1(\times 0((\mathbf{Y}\Gamma )(P0)))))) \\
        &\equiv \times 3(\times 2(\times 1(1))) \\
        &\equiv \times 3(\times 2(1)) \\
        &\equiv \times 3(2) \equiv 6
    \end{align*}

\end{frame}

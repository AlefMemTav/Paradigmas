\section{Recursão}

\begin{frame}[fragile]{Recursão}

    \begin{itemize}
        \item A recursão diz respeito a definição de uma função em termos de si mesma

        \item Ao contrário de outras notações, o cálculo $\lambda$ não permite esta definição
            diretamente, uma vez que os termos-$\lambda$ são anônimos

        \item Uma maneira de contornar isso é utilizar uma expressão-$\lambda$ que receba a 
            si mesma como argumento

        \item Além disso, é preciso lidar com os dois aspectos fundamentais de uma função
            recursiva: o(s) caso(s) base(s) e a chamada recursiva
    \end{itemize}

\end{frame}

\begin{frame}[fragile]{Estrutura básica da recursão}

    \begin{huge}
    \[
        \gamma(x) =  \left\lbrace \begin{array}{ll}
                    g(x), & \mbox{se}\ P(x), \\
                    h(x, \gamma), & \mbox{caso contrário}
                \end{array} \right.
    \]
    \end{huge}

    \begin{itemize}
        \item $P(x)$ é um predicado que retorna verdadeiro se $x$ é o valor que caracteriza um
            caso base

        \item Se $P(x)$ for verdadeiro, o valor de $\gamma$ em $x$ será dado pela função $g$

        \item Caso contrário, $\gamma(x)$ será dado por $h(x, \gamma)$, onde $h$ é uma função 
            que depende de $x$ e de $\gamma$
    \end{itemize}

\end{frame}

\begin{frame}[fragile]{Representação da estrutura básica da recursão no cálculo-$\lambda$}

    \begin{huge}
    \[
        \Gamma\equiv (\lambda \gamma x.(Px)(gx)(h))
    \]
    \end{huge}

    \begin{itemize}
        \item Observe que na definição da função recursiva $\Gamma$ é utilizado o 
            termo-$\lambda$ $I_F$

        \item Se o predicado $(Px)$ retornar verdadeiro, o retorno será o primeiro parâmetro 
            $(gx)$, que corresponde ao valor de $\Gamma$ para o caso base

        \item Se falso, será avaliada a função $h = h(x, \gamma)$

        \item Não há garantias, contudo, que $\Gamma\equiv \gamma$, pois no cálculo $\lambda$ 
            os termos são anônimos

        \item É preciso, portanto, definir um termo que garanta esta equivalência
    \end{itemize}
\end{frame}

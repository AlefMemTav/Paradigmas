\section{Recursão}

\begin{frame}[fragile]{Recursão}

    \begin{itemize}
        \item A recursão diz respeito a definição de uma função em termos de si mesma

        \item Ao contrário de outras notações, o cálculo $\lambda$ não permite esta definição
            diretamente, uma vez que os termos-$\lambda$ são anônimos

        \item Uma maneira de contornar isso é utilizar uma expressão-$\lambda$ que receba a 
            si mesma como argumento

        \item Além disso, é preciso lidar com os dois aspectos fundamentais de uma função
            recursiva: o(s) caso(s) base(s) e a chamada recursiva
    \end{itemize}

\end{frame}

\begin{frame}[fragile]{Estrutura básica da recursão}

    \begin{huge}
    \[
        f(x) =  \left\lbrace \begin{array}{ll}
                    g(x_0), & \mbox{se}\ P(x_0), \\
                    h(x, f(x)), & \mbox{caso contrário}
                \end{array} \right.
    \]
    \end{huge}

    \vspace{0.5in}

    $P(x)$ é um predicado que retorna verdadeiro se $x_0$ é o valor que caracteriza o caso base.
    Se $P(x)$ for verdadeira, o valor de $f$ em $x_0$ será dado pela função $g$; se $P(x)$ falsa, 
    $f(x)$ é igual a uma função $h$ que depende de $x$ e de $f(x)$.
\end{frame}

\begin{frame}[fragile]{Representação da estrutura básica da recursão no cálculo-$\lambda$}

    \begin{huge}
    \[
        F\equiv (\lambda fx.(Px)(gx)(hx(fx)))
    \]
    \end{huge}

    \vspace{0.3in}

    Observe que na definição da função recursiva $F$ é utilizado o termo-$\lambda$ $I_F$: se o
    predicado $(Px)$ retornar verdadeiro, o retorno será o primeiro parâmetro $(gx)$, que 
    corresponde ao valor de $F$ para o caso base; se falso, será avaliada a função $h$, que
    tem como parâmetros $x$ e $(fx)$.

    \vspace{0.1in}
    Não há garantias, contudo, que $F\equiv f$, pois no cálculo $\lambda$ os termos são anônimos.
    É preciso, portanto, definir um termo que garanta esta equivalência. 
\end{frame}

\section{Aritmética}

\begin{frame}[fragile]{Operadores aritméticos}

    \begin{itemize}
        \item Para computar expressões aritméticas, Prolog disponibiliza o predicado 
            pré-definido \code{prolog}{is}, cuja sintaxe é:

            \inputsyntax{prolog}{codes/is.pl}

        \item A variável \code{prolog}{X} é atada ao valor da expressão e é desatada no 
            \textit{backtracking}

        \item As expressões são semelhantes às utilizadas em outras linguagens

        \item Exemplos de expressões:

            \inputsyntax{prolog}{codes/expressions.pl}

        \item Parêntesis podem ser utilizado para evitar ambiguidades e alterar a ordem de 
            precedência dos operadores

            \inputsyntax{prolog}{codes/parenthesis.pl}

    \end{itemize}

\end{frame}

\begin{frame}[fragile]{Operadores relacionais}

    \begin{itemize}
        \item Para evitar que os operadores relacionais se assemelhem às setas, a ordem dos 
            símbolos é diferente do usual:

            \inputsyntax{prolog}{codes/relational.pl}

        \item Exemplos de regras baseadas em expressões aritméticas:

            \inputsyntax{prolog}{codes/juros.pl}

    \end{itemize}

\end{frame}

\begin{frame}[fragile]{Exemplo de uso de operadores ariméticos e relacionais}

    \inputsnippet{prolog}{1}{21}{codes/roots.pl}

\end{frame}

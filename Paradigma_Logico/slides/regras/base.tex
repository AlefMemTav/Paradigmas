\section{Manipulação da base de dados}

\begin{frame}[fragile]{Manipulação da base de dados}

    \begin{itemize}
        \item Prolog permite a manipulação direta da base de dados por meio de predicados 
            pré-definidos:

        \begin{enumerate}
            \item \code{prolog}{asserta(X)}: adiciona a cláusula \code{prolog}{X} como primeira 
            cláusula para o seu predicado. Como as rotinas de I/O, sempre falha no 
            \textit{backtracking} e não desfaz seu trabalho

            \item \code{prolog}{assertz(X)}: igual a anterior, mas adiciona como última 
                cláusula do predicado

            \item \code{prolog}{retract(X)}: remove a cláusula \code{prolog}{X} da base de dados
        \end{enumerate}

        \item Para remover uma clásula, é preciso marcar o predicado como dinâmico, antes da 
            definição do mesmo

        \item A sintaxe para tal é

            \inputsyntax{prolog}{codes/dynamic.pl}

    \end{itemize}

\end{frame}

\begin{frame}[fragile]{Prolog e variáveis globais}

    \begin{itemize}
        \item Não há variáveis globais em Prolog: as variáveis são locais às cláusulas

        \item A base de dados ``substitui'' as variáveis globais

        \item Ela permite que as cláusulas compartilhem informações entre si

        \item \code{prolog}{asserts} e \code{prolog}{retracts} são as ferramentas para 
            manipular os dados da base, dados estes que correpondem à variáveis globais

        \item Naturalmente, este recursos deve ser utilizado com parcimônio e cuidado, pois 
            ele modifica o estado do programa

        \item Alguns programadores tentam eliminar dados globais e o uso de 
            \code{prolog}{asserts} e \code{prolog}{retracts} em seus códigos Prolog
 
    \end{itemize}

\end{frame}

\begin{frame}[fragile]{Programas sem variáveis globais}

    \begin{itemize}

        \item É possível escrever programas que não modificam a base de dados, o que elimina o 
            problema das variáveis globais

        \item Isto pode ser feito passando as informações necessárias por meio dos argumentos 
            dos predicados

        \item A versão de \code{prolog}{assert} apresenta a seguir desfaz seu trabalho no 
            \code{prolog}{backtracking}:

            \inputsyntax{prolog}{codes/assert.pl}

        \item Inicialmente, a primeira cláusula é executada

        \item Se um objetivo posterior falhar, o \textit{backtracking} vai tentar a segunda 
            cláusula, que desfará o trabalho da primeira e falhar, resultando no efeito 
            desejado

    \end{itemize}

\end{frame}

\section{Estruturas de Dados}

\begin{frame}[fragile]{Estruturas de dados em Prolog}

    \begin{itemize}
        \item Uma estrutura de dados combina termos primitivos (átomos, inteiros, etc) e 
            estruturas em tipos compostos

        \item A sintaxe de declaração de uma estrutura de dados é 

            \inputsyntax{prolog}{codes/structures.pl}

        \item Cada argumento pode ser um tipo primitivo ou outra estrutura

        \item Sintaticamente, a declaração de uma estrutura é semelhante à declaração de um 
            fato ou regra

            \inputsyntax{prolog}{codes/cars.pl}

   \end{itemize}

\end{frame}

\begin{frame}[fragile]{Consultas envolvendo estruturas de dados}

    \begin{itemize}
        \item A ordem dos argumentos é importante nas consultas

            \inputsyntax{prolog}{codes/car_query.pl}

        \item Campos podem ser ignorados com a variável anônima

        \item Estruturas podem ser utilizadas em outras estruturas com o intuito de aumentar a 
            legibilidade

            \inputsyntax{prolog}{codes/car_struct.pl}

        \item O predicado \code{prolog}{not/1} recebe um objetivo como argumento e é bem 
            sucedida quando o objetivo falha, e falha quando o objetivo é bem sucedido

    \end{itemize}

\end{frame}

\begin{frame}[fragile]{Exemplo de estruturas recursivas}

    \inputsnippet{prolog}{1}{18}{codes/matrioskas.pl}

\end{frame}

\begin{frame}[fragile]{Exemplo de estruturas recursivas}

    \inputsnippet{prolog}{19}{38}{codes/matrioskas.pl}

\end{frame}

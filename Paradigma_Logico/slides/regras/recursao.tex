\section{Recursão}

\begin{frame}[fragile]{Recursão em Prolog}

    \begin{itemize}
        \item Em Prolog, a recursão acontece quando um predicado contém um objetivo que se 
            refere ao próprio predicado

        \item Como a cada consulta Prolog usa o corpo da regra para criar uma nova consulta 
            com novas variáveis, a recursão acontece naturalmente

        \item Uma chamada recursiva é composta por duas partes:

        \begin{enumerate}
            \item casos-base, e
            \item chamada recursiva
        \end{enumerate}

        \item Os casos-base são condições limítrofes (de contorno) que são sabidamente 
            verdadeiras

        \item O caso recursivo resolve o problema por meio de nova chamada da regra, em uma
            versão reduzida do problema

        \item A cada etapa da recursão, os casos-base são verificados: caso ocorram, a recursão 
            termina; caso contrário, a recursão continua

    \end{itemize}

\end{frame}

\begin{frame}[fragile]{Exemplo de recursão em Prolog}

    \inputsnippet{prolog}{1}{21}{codes/fact.pl}

\end{frame}

\begin{frame}[fragile]{Características da recursão em Prolog}

    \begin{itemize}
        \item O escopo das variáveis de uma regra é a própria regra

        \item Cada nível da recursão tem seu próprio conjunto de variáveis

        \item A unificação entre o objetivo e a cabeça cláusula forçam as relações entre as 
            variáveis de diferentes níveis

        \item Para garantir que os casos base sejam sempre testados, eles devem ser definidos 
            antes da chamada recursiva

        \item Cuidado: na cláusula correspondente à chamada recursiva, é preciso garantir que
            que os valores não correspondem ao casos bases

        \item Isto porque, devido ao fluxo de \textit{backtracking}, o retorno à porta
            \code{prolog}{redo} por meio de um ponto-e-vírgula pode fazer com que um valor
            que casas com um dos casos-base também seja testado na chamada recursiva!
            
        \item Importante: a ordem dos predicados na chamada recursiva pode afetar a 
            performance das consultas

    \end{itemize}

\end{frame}

\begin{frame}[fragile]{Exemplo de estruturas recursivas}

    \inputsnippet{prolog}{1}{18}{codes/matrioskas.pl}

\end{frame}

\begin{frame}[fragile]{Exemplo de estruturas recursivas}

    \inputsnippet{prolog}{19}{38}{codes/matrioskas.pl}

\end{frame}

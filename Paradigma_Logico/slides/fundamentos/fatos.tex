\section{Fatos}

\begin{frame}[fragile]{Fatos em Prolog}

    \begin{itemize}
        \item Os fatos são os predicados mais simples do Prolog

        \item Eles se assemelham a registros em um banco de dados relacional

        \item A sintaxe para a declaração de um fato é

            \inputsyntax{prolog}{codes/pred.pl}

        onde \code{prolog}{pred} é o nome do fato, e \code{prolog}{arg1, arg2, ..., argN}
        são os argumento, sendo $N$ a aridade

        \item O ponto final (`\texttt{.}') encerra todas as cláusulas de Prolog

        \item Se a aridade do predicado for zero, a sintaxe se reduz a

            \inputsyntax{prolog}{codes/pred_0.pl}

    \end{itemize}

\end{frame}

\begin{frame}[fragile]{Termos}

    \begin{itemize}
        \item Os argumentos podem ser quaisquer termos válidos de Prolog

        \item Os termos básicos do Prolog são

        \begin{itemize}
            \item \textbf{inteiro}: número positivo, negativo ou zero, com valor absoluto 
                máximo dependendo da implementação

            \item \textbf{átomo}: uma constante de texto iniciada com letra minúscula

            \item \textbf{variável}: começa com letra maiúscula ou sublinhado (`\texttt{\_}')

            \item \textbf{estrutura}: termos complexos
        \end{itemize}

        \item Algumas implementações podem estender esta lista, com strings e ponto flutuante, 
            por exemplo

        \item O uso de aspas simples permitem a construção de átomos por meio de qualquer 
            combinação válida de caracteres

        \item Os nomes dos predicados seguem as mesmas regras dos átomos
    \end{itemize}

\end{frame}

\begin{frame}[fragile]{Fatos em programas em Prolog}

    \begin{itemize}
        \item Os fatos são frequentemente utilizados para inserir informações no programa

        \item Por exemplo, para o predicato \code{prolog}{paciente/3} podem ser atestados os 
            seguintes fatos:

            \inputsyntax{prolog}{codes/paciente.pl}

        \item As aspas foram usadas nos nomes porque começam em maiúsculas e porque tem espaços

        \item Um interpretador Prolog deve fornecer meios de inserção de fatos e regras em uma 
            base de dados dinâmica, a qual pode ser consultada

        \item A base de dados é atualizada por meio de consultas (\code{prolog}{consult/1} ou
            \code{prolog}{reconsult/1}

    \end{itemize}

\end{frame}

\begin{frame}[fragile]{Fatos em programas em Prolog}

    \begin{itemize}
        \item Os predicados podem ser inseridos diretamente no interpretador, mas não são
            gravados entre as sessões

        \item Isto pode ser feito por meio do predicados \code{prolog}{asserta/1} e
            \code{prolog}{assertz/1}

        \item O primeiro insere um novo fato como primeiro dentre os fatos declarados para
            o predicado

        \item O segundo insere o novo fato como o último dentre os já declarados

        \item Os nomes utilizados no fato são indiferentes para o Prolog, mas para a aplicação 
            as relações devem ser compatíveis com a semântica dos identificadores escolhidos

        \item Prolog considera distintos os fatos \code{prolog}{fato(a, b)} e 
            \code{prolog}{fato(b, a)}



    \end{itemize}

\end{frame}

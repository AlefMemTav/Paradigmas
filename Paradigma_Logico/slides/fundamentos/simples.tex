\section{Consultas simples}

\begin{frame}[fragile]{Consultas simples}

    \begin{itemize}
        \item Uma vez que a base de dados do interpretador Prolog seja alimentado com fatos, o
            este interpretador pode responder a consultas (\textit{queries}) a respeito dos
            fatos

        \item As consultas em Prolog funcionam por meio do casamento de padrões
            (\textit{pattern matching})

        \item O padrão de uma consulta é denominado \textbf{objetivo} (\textit{goal})

        \item Se algum fato atinge o objetivo, a consulta é bem sucedida e o interpretador
            responde ``\texttt{Sim}'' (\code{prolog}{true.})

        \item Caso contrário, a consulta falha e o interpretador responde ``\texttt{Não}''
            (\code{prolog}{false.})

        \item O casamento de padrões do Prolog é denominado \textbf{unificação}

    \end{itemize}

\end{frame}

\begin{frame}[fragile]{Unificação}

    \begin{block}{Unificação (versão simplificada)}
        Se o programa contém apenas fatos, a unificação é bem sucedida se as três condições
            abaixo são satisfeitas:
        \begin{enumerate}
            \item o predicado citado no objetivo e na base de dados é o mesmo,
            \item ambos tem a mesma aridade,
            \item todos os argumentos são os mesmos.
        \end{enumerate}

    \end{block}

\end{frame}

\begin{frame}[fragile]{Exemplos de unificação}

    \inputsnippet{prolog}{1}{21}{codes/matching.pl}

\end{frame}

\begin{frame}[fragile]{Variáveis lógicas}

    \begin{itemize}
        \item Objetivos podem ser generalizados por meio de variáveis do Prolog

        \item Estas variáveis não se comportam como variáveis em outras linguagens, e são 
            denominadas variáveis lógicas

        \item Elas substituem um ou mais argumentos no objetivo

        \item O nome e a aridade do predicado devem permanecer os mesmos, porém o argumento 
            substituído pela variável casará positivamente com qualquer termo

        \item Após uma unificação bem sucedida, a variável terá como valor os termos os quais 
            casaram com ela

        \item Este processo ata (\textit{to bind}) os valores (termos) à variável

        \item Estes valores serão o retorno da consulta

    \end{itemize}

\end{frame}

\begin{frame}[fragile]{Variáveis lógicas}

    \begin{itemize}
        \item Um valor será retornado: caso mais de um valor seja casado com a variável, 
            Prolog fornece mecanismos para a extração de todos eles

        \item Após uma resposta pode ser inserido o ponto-e-vírgula (`\texttt{;}'): isto faz 
            com que Prolog procure por casamentos alternativos para as variáveis

        \item Qualquer outra entrada encerra a consulta

        \item Se não houveram mais alternativas, o retorno será ``\texttt{Não}'' 

        \item Os nomes das variáveis devem iniciar em maiúsculas

        \item Todos os argumentos da consulta podem ser substituídos por variáveis

    \end{itemize}

\end{frame}

\begin{frame}[fragile]{Exemplo de uso de variáveis lógicas}

    \inputsnippet{prolog}{1}{21}{codes/variables.pl}

\end{frame}

\begin{frame}[fragile]{\it Backtracking}

    \begin{itemize}
        \item Quando Prolog tenta satisfazer um objetivo a respeito de um predicado, ele 
            percorre todas as cláusulas que definem o predicado

        \item Uma vez que há um casamento (entre argumentos ou variáveis), a cláusula em 
            questão é marcada, indicando que ela satisfez o objetivo

        \item  O interpretador retorna o valor deste casamento, no caso de variáveis, ou 
            ``\texttt{Sim}'', no caso de argumentos

        \item Se o usuário solicitar mais respostas, ele retoma a busca entre as cláusulas a 
            partir do ponto em que ele parou

        \item Nesta retomada a última variável que foi casada é desatada 

        \item Este processo é denominado \textit{backtracking}

    \end{itemize}

\end{frame}

\begin{frame}[fragile]{Objetivos e fluxo de controle}

    \begin{itemize}
        \item Um objetivo em Prolog tem quatro portas que representam o fluxo de controle
            quando este avalia o objetivo: chamada 
            (\textit{call}), saída (\textit{exit}), retomada (\textit{redo}) e falha 
            (\textit{fail})

        \item A avaliação do objetivo inicia na chamada

        \item Se a chamada é bem sucedida (há um casamento de padrão), a avaliação saí 
            (encerra, \textit{exit}), atando as variáveis com os resultados obtidos

        \item Caso contrário, a avaliação falha: não há mais valores que possam satisfazer o 
            objetivo

        \item Se bem sucedida e é seguida de um ';', a avaliação é retomada (\textit{redo}) a
            partir do ponto que parou 

        \item As variáveis são desatadas de seus valores e se busca por novos valores que 
            também satisfaçam o objetivo
    \end{itemize}

\end{frame}

\begin{frame}[fragile]{Visualização do fluxo em Prolog}

    \begin{figure}[h]
        \centering
        \begin{tikzpicture}
            \draw[thick] (0, 0) rectangle (5, 5);

            \draw[-latex] (-1, 4) node[anchor=east] {\it call} -- (0, 4);
            \draw[latex-] (-1, 1) node[anchor=east] {\it fail} -- (0, 1);
            \draw[latex-] (6, 4) node[anchor=west] {\it exit} -- (5, 4);
            \draw[-latex] (6, 1) node[anchor=west] {\it redo} -- (5, 1);

            \node at (2.5, 2.5) { Objetivo };

        \end{tikzpicture}
    \end{figure}

\end{frame}


\begin{frame}[fragile]{Rastreamento}

    \begin{itemize}
        \item É possível rastrear (\textit{to trace}) o fluxo de controle do Prolog por meio
            do predicado \code{prolog}{trace/0}

        \item Todas as consultas subsequentes serão rastreadas

        \item O comando \code{prolog}{creep} avança para o próximo passo da execução, de modo
            semelhante ao comando \textit{step} dos depuradores das linguagens procedurais

        \item No SWI-Prolog, este comando pode ser inserido através da barra de espaço

        \item Para desativar o modo \code{prolog}{trace}, use o predicado 
            \code{prolog}{notrace/0}, seguido do predicado \code{prolog}{nodebug/0}

    \end{itemize}

\end{frame}


\begin{frame}[fragile]{Generalizações}

    \begin{itemize}
        \item As variáveis lógicas permite a declaração de generalizações

        \item Generalizações são fatos que casam com qualquer outro valor 

        \item Por exemplo, a afirmação ``Todos dormem'' pode ser declarada como

            \inputsyntax{prolog}{codes/sleeps.pl}

        \item Outro fato importante relativo às variáveis lógicas é que se uma mesma variável
            é utilizada em vários argumentos, o casamento só será bem sucedido se esta variável
            casar com o mesmo valor em todas as suas ocorrências
    \end{itemize}

\end{frame}

\begin{frame}[fragile]{Exemplo de mesma variável lógica em múltiplos argumentos}

    \inputsnippet{prolog}{1}{21}{codes/same_var.pl}

\end{frame}

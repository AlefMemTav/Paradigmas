\section{Conceitos elementares}

\begin{frame}[fragile]{Produto Cartesiano}

    \begin{block}{Produto Cartesiano}
        Sejam $A$ e $B$ dois conjuntos. O produto cartesiano $A\times B$ é o conjunto de todos
        os pares ordenados cujo primeiro componente é um elemento de $A$ e o segundo componente
        é um elemento de $B$, isto é,
        \[
            A\times B = \lbrace (a, b)\ |\ a\in A, b\in B\rbrace
        \]
    \end{block}

\end{frame}

\begin{frame}[fragile]{Exemplos de produtos cartesianos}

    \begin{enumerate}
        \item Seja $A = \lbrace 1, 2, 3\rbrace$ e $B = \lbrace a, b\rbrace$. Então
        \[
            A\times B = \lbrace (1, a), (1, b), (2, a), (2, b), (3, a), (3, b)\rbrace
        \]
        e
        \[
            B\times A = \lbrace (a, 1), (a, 2), (a, 3), (b, 1), (b, 2), (b, 3)\rbrace
        \]

        \item Seja $C$ o conjunto dos times que participam de um campeonato de futebol. A
            tabela $T$ dos jogos da primeira fase do campeonato, onde cada time enfrenta todos os
            outros em jogos de ida e volta é o conjunto
        \[
            T = \lbrace (a, b)\in C\times C\ |\ a\neq b\rbrace
        \]

        \item $\mathbb{R}^2 = \mathbb{R}\times \mathbb{R}$
    \end{enumerate}

\end{frame}

\begin{frame}[fragile]{Relações e Funções}

    \begin{block}{Relação de $A$ em $B$}
        Sejam $A$ e $B$ dois conjuntos. Uma \textbf{relação} $R$ de $A$ em $B$ é um subconjunto 
            $R\subset A\times B$.
    \end{block}

    \vspace{0.1in}

    \begin{block}{Função de $A$ em $B$}
        Uma relação $f$ de $A$ em $B$ é uma \textbf{função} de $A$ em $B$ se, para qualquer 
            $a\in A$, existe um único $b\in B$ tal que $(a, b)\in A\times B$. 

        Notação: $f: A\to B$
    \end{block}

    \vspace{0.1in}

    \textbf{Observação}: se $f$ é uma função de $A$ em $B$, então $(a, b)\in f$ pode ser 
        escrito como $f(a) = b$.
        
\end{frame}

\begin{frame}[fragile]{Domínio, imagem e gráfico}

    \begin{block}{Domínio e imagem de uma função $f$ de $A$ em $B$}
        Seja $f$ uma função de $A$ em $B$. O conjunto $A$ é denominado \textbf{domínio} da
            função $f$, e o conjunto $B$ o \textbf{contradomínio} de $f$. Além disso, o conjunto
        \[
            Img(f) = \lbrace b \in B\ |\ \exists\, a\in A\ \mbox{tal que}\ f(a) = b \rbrace
        \]
        é a \textbf{imagem} da função $f$. Outra notação comum para o conjunto imagem de $f$ é 
        $f(A)$.
    \end{block}

    \vspace{0.05in}

    \begin{block}{Gráfico de uma função}
        Seja $f$ uma função de $A$ em $B$. O gráfico de $f$ é o conjunto
        \[
            Gr(f) = \lbrace (x, f(x))\ |\ x\in A\rbrace
        \]
    \end{block}

\end{frame}

%%
% - Exemplos de funções
% - Exemplos de domínio e imagem?
% - Propriedades de funções
% - Operações em funções (composição, inversas)

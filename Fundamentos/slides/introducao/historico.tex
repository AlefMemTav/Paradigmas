\section{Contexto Histórico}

\begin{frame}[fragile]{Os ideais de Leibniz}

    O matemático alemão Gottfried Wilhelm Leibniz (1646-1716) tinha dois ideais:

    \vspace{0.2in}

    \begin{enumerate}
        \item Criar uma ``\textit{linguagem universal}'' na qual todos os problemas pudessem
            ser descritos
        \item Encontrar um método de decisão para todos os problemas descritos nesta linguagem
            pudessem ser resolvidos
    \end{enumerate}

\end{frame}

\begin{frame}[fragile]{Teoria dos conjuntos e lógica de primeira ordem}

    No que diz respeito aos problemas matemáticos, o primeiro ideal de Leibniz pode ser alcançado       por meio de uma Teoria de Conjuntos formulada em termos de uma Lógica de Primeira Ordem. 

    \vspace{0.2in}

    O matemático/lógico/filófoso inglês Bertrand Russel (1872-1970) e o matemátco/lógico alemão
    Ernst Zermelo (1871-1953) trouxeram grandes contribuições para esta questão.

\end{frame}

\begin{frame}[fragile]{A grande questão}

    O segundo ideal trazia consigo uma grande questão filosófica, que ficou conhecidade como
    \textit{Entscheidungsproblem}:

    \vspace{0.2in}
    \begin{center}
        {\it \Large É possível resolver todos os problemas descritos na linguagem universal?}
    \end{center}

\end{frame}

\begin{frame}[fragile]{Turin e Church}

    A resposta negativa para o \textit{Entscheidungsproblem} foi dada em 1936, independentemente,
    por dois grandes matemáticos. Para tal, ele precisaram formalizar a noção de 
    decidibilidade, ou computabilidade:

    \vspace{0.2in}

    \begin{itemize}
        \item Alonzo Church (1936) inventou um sistema formal, denominado Cálculo $\lambda$,
            e definiu a noção de função computável por meio deste sistema
        \item Alan Turin (1936/7) criou as Máquinas de Turin, e definiu computabilidade em termos
            destas máquinas
    \end{itemize}

    \vspace{0.2in}

    Características de ambos modelos estiveram presentes nas diversas linguagens de programação
    ao longo da história.

\end{frame}

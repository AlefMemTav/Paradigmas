\section{Variáveis}

\begin{frame}[fragile]{Variáveis e Assembly}

    \begin{itemize}
        \item Como a programação imperativa consiste em uma série de comandos a serem executados,
            é preciso manter registro de tudo que já foi computado

        \item As variáveis mantém o estado do programa

        \item Elas permitem que o controle decida o próximo estado a ser atingido, após a 
            modificação do estado atual

        \item Em Assembly as variáveis correspondem aos registradores e aos endereços de memória
            acessíveis

        \item Os nomes dos registradores já estão pré-definidos

        \item Já as áreas de memória acessíveis são identificadores por \textbf{rótulos}, e cada
            \textit{byte} desta memória deve ser acessado por meio de seu endereço

        \item Cada rótulo corresponde à área de memória onde está a instrução que o sucede
    \end{itemize}

\end{frame}

\begin{frame}[fragile]{Estrutura dos programas em Assembly}

    \begin{itemize}
        \item Embora cada arquitetura tenha uma linguagem Assembly distinta, a estrutura dos 
            programas escritos em Assembly tende a ser a mesma

        \item Esta estrutura é denominada \textbf{formato de quatro campos}

            \inputsyntax{nasm}{codes/4fields.s}

        \item A primeira coluna contém os \textbf{rótulos} do programa: cada rótulo é seguido de 
            dois pontos

        \item A segunda coluna contém os \textbf{mnemônicos} que correspondem às instruções a serem
            executadas

        \item A terceira coluna contém os \textbf{operandos} de cada instrução, se houverem

        \item A quarta coluna contém os \textbf{comentários} pertinentes
    \end{itemize}

\end{frame}

\begin{frame}[fragile]{Seções}

    \begin{itemize}
        \item Além das quatro colunas identificadas, os códigos Assembly também são divididos
            em \textbf{seções}

        \item A seção \texttt{.text} contém as instruções do programa

        \item A seção \texttt{.data} contém as variáveis inicializadas

        \item Cada variável da seção \texttt{.data} é associada a um nome, que será
            acessível em toda seção \texttt{.text}

        \item As pseudo-instruções \code{nasm}{DB}, \code{nasm}{DW} e \code{nasm}{DD} 
            (\textit{define byte, define word} e \textit{define double word}) são 
            utilizadas para declarar variáveis cuja unidade de medida é 1, 2, ou 4 \textit{bytes},
            respectivamente

        \item Um vetor de variáveis pode ser inicializado por meio de vírgulas

            \inputsyntax{nasm}{codes/array.st}

        \item Um vetor com $25$ zeros pode ser inicializado com a sintaxe

            \inputsyntax{nasm}{codes/narray.st}

    \end{itemize}

\end{frame}

\begin{frame}[fragile]{Seções}

    \begin{itemize}
        \item A seção \texttt{.bss} (\textit{Block Started by Symbol}) é utilizada para reservar
            memória para variáveis não inicializadas

        \item A pseudo-instrução \code{nasm}{RESB} (\textit{reserve byte}), cuja sintaxe é

            \inputsyntax{nasm}{codes/resb.st}

        reserva \texttt{imm} \textit{bytes} de memória

        \item As instruções \code{nasm}{RESW} e \code{nasm}{RESD} tem mesma sintaxe, mas
            as unidades de medida correspondem àquelas apresentadas para as instruções
            \code{nasm}{DW} e \code{nasm}{DD}, respectivamente

        \item A palavra-chave \code{nasm}{global} declara os símbolos que serão exportados pelo
            código-objeto

        \item O símbolo \code{nasm}{_start} será ponto de partida da execução de programas
            escritos em Assembly e montados com o NASM
    \end{itemize}

\end{frame}

\begin{frame}[fragile]{Tipos de dados em Assembly}

    \begin{itemize}
        \item As linguagens Assembly não mantém registro dos tipos das variáveis do programa

        \item O significado dos padrões binários armazenados nos registradores ou na memória
            fica a cargo do programador, que deve manter a coerência dos mesmos ao longo do
            programa

        \item Por exemplo, as instruções de adição não fazem distinção se os números são
            sinalizados ou não

        \item O \textit{assembler} utilizado pode fornecer macros para facilitar a inicialização
            de variáveis, como no caso dos valores imediatos e de strings

        \item Mesmo com estas facilidades, não há nenhum verificação, nem em tempo de compilação,
            nem em tempo de execução, de tipos de dado, limites, violações, etc
    \end{itemize}

\end{frame}

\begin{frame}[fragile]{Hello World!}
    \inputsnippet{nasm}{1}{22}{codes/hello.s}
\end{frame}

\begin{frame}[fragile]{Hello World com entrada de usuário}
    \inputsnippet{nasm}{1}{22}{codes/hello2.s}
\end{frame}

\begin{frame}[fragile]{Hello World com entrada de usuário}
    \inputsnippet{nasm}{23}{42}{codes/hello2.s}
\end{frame}

\begin{frame}[fragile]{Variáveis e memória}

    \begin{itemize}
        \item É possível manipular a memória por meio de atribuições

        \item Para tal, é preciso carregar a informação para um registrador, realizar o
            processamento desejado e depois mover o resultado para a memória

        \item A sintaxe para indicar um endereço de memória é \texttt{[X]}, onde 
            \texttt{X} é um rótulo ou um registrador

        \item É possível realizar aritmética de ponteiros, através da sintaxe
        \begin{center}
            \texttt{[reg + scale*reg + offset]}
        \end{center}

        \item Os valores possíveis para o fator de escala \texttt{scale} são: 1, 2, 4 ou 8

        \item O primeiro registrador é denominado \textbf{registrador base}

        \item O registrador multiplicado pelo fator de escala é denominado \textbf{registrador 
            índice}

        \item Tanto o registrador índice como o deslocamento (\texttt{offset}) podem ser omitidos
    \end{itemize}

\end{frame}

\begin{frame}[fragile]{Exemplo de manipulação de variável em memória}
    \inputsnippet{nasm}{1}{22}{codes/variables.s}
\end{frame}

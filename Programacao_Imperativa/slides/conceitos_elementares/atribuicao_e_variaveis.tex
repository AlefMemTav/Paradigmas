\section{Atribuição e Variáveis}

\begin{frame}[fragile]{Atribuições}

    \begin{itemize}
        \item Atribuições mudam os valores de uma dada localização

        \item Em Assembly, atribuições são feitas por meio do comando \code{nasm}{MOV}:

            \inputsyntax{nasm}{codes/mov.st}

        \item A sintaxe Assembly para comandos com dois parâmetros é a seguinte:
    
            \inputsyntax{nasm}{codes/command.st}

        onde \texttt{dest} é a localização onde será escrito o valor contido em \texttt{orig}

        \item Na segunda forma do comando \code{nasm}{MOV}, \texttt{imm} refere-se a um valor
            \textbf{imediato}

        \item Esta valor corresponde a um número inteiro, em notação decimal ou hexadecimal (por
            meio do sufixo \texttt{H})

    \end{itemize}

\end{frame}

\begin{frame}[fragile]{Exemplo de atribuição}
    \inputsnippet{nasm}{1}{22}{codes/atribuicao.s}
\end{frame}

\begin{frame}[fragile]{Adição e subtração}

    \begin{itemize}
        \item O valor a ser atribuído pode ser o resultado de uma das quatro operações
            aritméticas

        \item A \textbf{adição} e a \textbf{subtração} tem a mesma sintaxe:
            \inputsyntax{nasm}{codes/add_sub.st}

        \item Na primeira forma, o valor armazenado em \texttt{orig} é adicionado/subtraído do
            valor contido em \texttt{dest}, e o resultado é armazenado em \texttt{dest}

        \item Na segunda forma, o valor do registrador é atualizado, através da adição/subtração
            do valor imediato

        \item Ao contrário da matemática, nenhum dos dois comandos é comutativo
    \end{itemize}

\end{frame}

\begin{frame}[fragile]{Exemplo de aplicação da adição e da subtração}
    \inputsnippet{nasm}{1}{19}{codes/euler.s}
\end{frame}

\begin{frame}[fragile]{Multiplicação}

    \begin{itemize}
        \item A multiplicação não compartilha da mesma sintaxe da adição e da subtração

        \item Isto porque, ao multiplicar dois números de $b$-\textit{bits}, o resultado será um
            número de $2b$-\textit{bits}

        \item Assim, a sintaxe da multiplicação é

            \inputsyntax{nasm}{codes/mult.st}

        \item Se \texttt{reg} é um registrador de 8-\textit{bits}, este será multiplicado por
            \code{nasm}{AL} e o produto será armazenado em \code{nasm}{AX}

        \item Por exemplo,

            \inputsyntax{nasm}{codes/mult8.st}

        equivale a \code{asm}{AX} = \code{asm}{AL}$\cdot$\code{asm}{BH}
    \end{itemize}

\end{frame}

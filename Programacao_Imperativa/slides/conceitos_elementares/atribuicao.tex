\section{Atribuições e Operadores}

\begin{frame}[fragile]{Atribuições}

    \begin{itemize}
        \item Atribuições mudam os valores de uma dada localização

        \item Em Assembly, atribuições são feitas por meio do comando \code{nasm}{MOV}:

            \inputsyntax{nasm}{codes/mov.st}

        \item A sintaxe Assembly para comandos com dois parâmetros é a seguinte:
    
            \inputsyntax{nasm}{codes/command.st}

        onde \texttt{dest} é a localização onde será escrito o valor contido em \texttt{orig}

        \item Na segunda forma do comando \code{nasm}{MOV}, \texttt{imm} refere-se a um valor
            \textbf{imediato}

        \item Esta valor corresponde a um número inteiro, em notação decimal ou hexadecimal (por
            meio dos sufixos \texttt{H} ou \texttt{h})

    \end{itemize}

\end{frame}

\begin{frame}[fragile]{Exemplo de atribuição}
    \inputsnippet{nasm}{1}{22}{codes/atribuicao.s}
\end{frame}

\begin{frame}[fragile]{Adição e subtração}

    \begin{itemize}
        \item O valor a ser atribuído pode ser o resultado de uma das quatro operações
            aritméticas

        \item A \textbf{adição} e a \textbf{subtração} tem a mesma sintaxe:
            \inputsyntax{nasm}{codes/add_sub.st}

        \item Na primeira forma, o valor armazenado em \texttt{orig} é adicionado/subtraído do
            valor contido em \texttt{dest}, e o resultado é armazenado em \texttt{dest}

        \item Na segunda forma, o valor do registrador é atualizado, através da adição/subtração
            do valor imediato

        \item Ao contrário da matemática, nenhum dos dois comandos é comutativo
    \end{itemize}

\end{frame}

\begin{frame}[fragile]{Exemplo de aplicação da adição e da subtração}
    \inputsnippet{nasm}{1}{19}{codes/euler.s}
\end{frame}

\begin{frame}[fragile]{Multiplicação}

    \begin{itemize}
        \item A multiplicação não compartilha da mesma sintaxe da adição e da subtração

        \item Isto porque, ao multiplicar dois números de $b$-\textit{bits}, o resultado será um
            número de $2b$-\textit{bits}

        \item Assim, a sintaxe da multiplicação é

            \inputsyntax{nasm}{codes/mult.st}

        \item Se \texttt{reg} é um registrador de 8-\textit{bits}, este será multiplicado por
            \code{nasm}{AL} e o produto será armazenado em \code{nasm}{AX}

        \item Por exemplo,

            \inputsyntax{nasm}{codes/mult8.st}

        equivale a \code{nasm}{AX} = \code{asm}{AL}$\cdot$\code{asm}{BH}
    \end{itemize}

\end{frame}

\begin{frame}[fragile]{Multiplicação}

    \begin{itemize}
        \item Se \texttt{reg} é um registrador de 16-\textit{bits}, este será multiplicado por
            \code{nasm}{AX} e o produto será armazenado no par de registradores 
            \code{nasm}{DX}:\code{nasm}{AX}

        \item \code{nasm}{AX} conterá os \textit{bits} menos significativos do resultado, e 
            os mais significativos serão escritos em \code{nasm}{DX}

        \item No caso de um registrador de 32-\textit{bits}, serão necessários dois registradores
            de igual tamanho para armazenar o resultado

        \item Por exemplo,

            \inputsyntax{nasm}{codes/mult8.st}

        equivale a \code{nasm}{EDX}:\code{nasm}{EAX} = \code{asm}{EAX}$\cdot$\code{asm}{EBX}

        \item A instrução \code{asm}{MUL} não tem suporte para valores imediatos
    \end{itemize}
\end{frame}

\begin{frame}[fragile]{Exemplo de uso da multiplicação}
    \inputsnippet{nasm}{1}{19}{codes/area.s}
\end{frame}

\begin{frame}[fragile]{Divisão}

    \begin{itemize}
        \item A sintaxe da instrução \code{nasm}{DIV} é semelhante à da instrução \code{nasm}{MUL}

            \inputsyntax{nasm}{codes/div.st}

        \item Pela Divisão de Euclides, dados dois inteiros $a, b$, existem únicos $q, r$ tais
        que
        \[
            a = bq + r, \ \ \ \ 0 \leq r < |b|
        \]

        \item $a$ é o \textbf{dividendo}, $b$ o \textbf{divisor}, $q$ o \textbf{quociente} e 
            $r$ o \textbf{resto} da divisão

        \item A divisão por zero leva a uma exceção

        \item Se o quociente for maior do que o registrador reservado para armazená-lo, ocorrerá um
            erro de \textit{overflow}
    \end{itemize}

    \begin{table}[ht]
        \centering

        \begin{tabular}{lrrr}
            \toprule
            \textbf{Divisor} (\texttt{reg}) & \textbf{Dividendo} & \textbf{Quociente} & 
                \textbf{Resto} \\
            \midrule
                8-\textit{bits} & \texttt{AX} & \texttt{AL} & \texttt{AH} \\
                16-\textit{bits} & \texttt{DX:AX} & \texttt{AX} & \texttt{DX} \\
                32-\textit{bits} & \texttt{EDX:EAX} & \texttt{EAX} & \texttt{EDX} \\
            \bottomrule
        \end{tabular}
        
    \end{table}

\end{frame}

\begin{frame}[fragile]{Exemplo de divisão}
    \inputsnippet{nasm}{1}{22}{codes/imc.s}
\end{frame}

\section{Conceitos Elementares}

\begin{frame}[fragile]{Programação Imperativa}

    \begin{itemize}
        \item É uma abstração de computadores reais, os quais são baseados em máquinas de Turing e 
            na arquitetura de Von Neumann, com registradores e memória

        \item O conceito fundamental é o de estados modificáveis

        \item Variáveis e atribuições são os construtos de programação análogos aos registradores 
            dos ábacos e dos quadrados das máquinas de Turing

        \item Linguagens imperativas fornecem uma série de comandos que permitem a manipulação do 
            estado da máquina
    \end{itemize}

\end{frame}

\begin{frame}[fragile]{Estados}

    \begin{itemize}
        \item \textbf{Nomes} podem ser associados a um valor e depois serem associados a 
            um outro valor distinto

        \item O conjunto de nomes, de valores associados a estes nomes e a localização do ponto 
            de controle do programa constituem o \textbf{estado} do programa

        \item Um programa em execução gera uma sequência de estados 

        \item As \textbf{transições} entre os estados é feita por meio de atribuições e comandos 
            de sequenciamento

        \item A menos que sejam cuidadosamente escritos, programas imperativos só podem ser 
            entendidos em termos de seu comportamento de execução

        \item Isto porque, durante a execução, qualquer variável pode ser referenciada, o 
            controle pode se mover para um ponto arbitrário e qualquer valor pode ser modificado

    \end{itemize}

\end{frame}

\begin{frame}[fragile]{Valores, variáveis e nomes}

    \begin{itemize}
        \item Os padrões binários que o hardware reconhece são considerados, nas linguagens de 
            programação, \textbf{valores}

        \item Uma unidade de armazenamento em hardware equivale a uma \textbf{variável}, no ponto 
            de vista do programa

        \item O endereço da unidade de armazenamento é interpretado como um \textbf{nome} no 
            contexto de programação

        \item Assim, um nome é associado tanto ao endereço (localização) de uma unidade de 
            armazenamento quanto ao valor armazenado nesta unidade

        \item A localização é denominada \textit{l-value}, e o valor em si, \textit{r-value}

        \item Por exemplo, na expressão:
        \[
            x = x + 2;
        \]
        o $x$ à esquerda da expressão é um \textit{l-value} (localização), enquanto o $x$ da 
        direita é um \textit{r-value} (valor)

    \end{itemize}

\end{frame}

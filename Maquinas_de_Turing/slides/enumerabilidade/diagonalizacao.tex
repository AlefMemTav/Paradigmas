\section{Diagonalização}

\begin{frame}[fragile]{Exemplo de conjunto não-enumerável}

    \begin{block}{Teorema de Cantor}
        O conjunto de todos os conjuntos de inteiros positivos não é enumerável.
    \end{block}

\end{frame}

\begin{frame}[fragile]{Demonstração do Teorema de Cantor}

    \begin{itemize}
        \item Uma forma de demonstrar o Teorema de Cantor é utilizar uma técnica chamada
            diagonalização

        \item A ideia é, a partir de uma lista $L$ de conjuntos de inteiros positivos,
            construir um conjunto $\Delta(L)$ de inteiros positivos que não pertence à lista
            $L$

        \item Caso este conjunto seja acrescido à lista $L$, é possível aplicar a mesma
            técnica para construir um novo conjunto $\Delta(L^*)$ que não pertence à
            lista $L^* = L\cup \Delta(L)$

        \item Assim, não existe nenhuma lista que enumera o conjunto de todos os conjuntos de
            inteiros positivos
    \end{itemize}

\end{frame}

\begin{frame}[fragile]{Construção do conjunto $\Delta(L)$}

    \begin{itemize}
        \item Seja $L = S_1, S_2, S_3, \ldots$ uma lista de conjuntos de inteiros positivos $S_i$,
            $i\in \mathbb{Z}^+$

        \item Defina
        \[
            \Delta(L) = \lbrace n\in\mathbb{Z}^+\ |\ n \not\in S_n\rbrace
        \]

        \item Da definição acima, $\Delta(L)\subset \mathbb{Z}^+$

        \item Como sugerido pela própria notação, o conjunto $\Delta(L)$ depende da lista $L$:
            cada lista $L_j$ de conjuntos de inteiros positivos gera um conjunto $\Delta(L_j)$
            em particular

        \item Por exemplo, se $L = S_1, S_2, S_3, \ldots$ é tal que $S_i$ é o conjunto dos
            $i$ primeiros números primos, temos que
        \[
            \Delta(L) = \lbrace 1, 4, 6, 8, 9, 10, \ldots \rbrace
        \]
    \end{itemize}

\end{frame}

\begin{frame}[fragile]{Demonstração por contradição}

    \begin{itemize}
        \item Suponha, por contradição, que $\Delta(L)\in L$, isto é, que o conjunto $\Delta(L)$
            seja listado, em algum momento, por $L$

        \item Assim, existe um $k\in\mathbb{Z}^+$ tal que $\Delta(L) = S_k$

        \item Em relação ao inteiro positivo $k$ há dois cenários possíveis

        \item Se $k\in S_k$, então $\Delta(L)\neq S_k$, pois por definição, se $k\in\Delta(L)$ então
            $k\not\in S_k$

        \item Logo, deveríamos ter $k\not\in S_k$ mas, neste caso, teríamos que ter $k\in\Delta(L)$,
            de modo que $S_k\neq \Delta(L)$

        \item Portanto, a hipótese de que $\Delta(L)\in L$ leva a contradição
            $(\Delta(L)=S_k) \land (\Delta(L)\neq S_k)$

        \item Assim, $\Delta(L)\not\in L$, completando a demonstração do Teorema de Cantor
    \end{itemize}

\end{frame}

\begin{frame}[fragile]{Corolário do Teorema de Cantor}

    \begin{block}{Corolário}
        O conjunto $\mathbb{R}$ dos números reais não é enumerável.
    \end{block}

    \begin{block}{Demonstração}
        Seja $x\in\mathbb{R}$ tal que $0 < x < 1$. Então $x$ tem uma expansão decimal da forma
        \[
            0,x_1x_2x_3\ldots
        \]
        Seja $S_x$ o conjunto de inteiros positivos tal que $n\in S_x$ se, e somente se, 
            $x_n = 1$. Deste modo, qualquer conjunto $S\subset \mathbb{Z}^+$ está associado a,
            pelo menos, um número real $y\in (0, 1)$. Assim, se os reais fossem enumeráveis,
            o conjunto de todos os conjuntos de inteiros positivos seria enumerável, contradizendo
            o Teorema de Cantor.
            
    \end{block}
\end{frame}
